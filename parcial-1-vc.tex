 
% \newpage
%%\clearpage
%\pagenumbering{arabic}
%%\setcounter{page}{13}
% 
% 
%%\addcontentsline{toc}{chapter}{ \hfill 12}
% 
% \addtocontents{toc}{\protect\contentsline{chapter}{CAPÍTULO I. Generalidades del diseño térmico y cámaras de refrigeración de insulina \hfill 12}{}{}}
% 


%Gemas de Galculo vectored "Paro e Paso"

En cursos anteriores, se habian manejado funclones de das variables $x$ y $y$ es decir funciones en donde $x$ denotaba, a la variable independiente que censtitióa el dominio de la función $y=f(x)$, asi uamos a tener una waciable y que entonces so wa a poder establecer a partio del cualor dex. Este tipo de expresiones las uamos a xepresentar en un espacio de 2 dimensiones (2D) o tambiém le notado $2 R$, como seilus tra.
2))
Figura 1.1 .1

Delamisma manera podemes denotar un espacio de una
sola dimensión, como enel caso de la recta numérica, en la cvalen el curso de habiamos definido Respectivamente distintos tipos de interiealos, con el fin de establecer los walores en los cuales una función estaba ono, definida, as í teníamos
$\stackrel{a}{2}, \quad \longrightarrow$ intercealo cerradopor ambes lados: $a \leq x \leq b$
intervalo abierto por les dos Lados $a<x<b$
Intercalo abierto por el lada izquierdo $a<x \leq b$
interuala abierto par ol todo derecho $a \leq x<b$
%Escaneado con CamScanner


%%%%%%%%%%%%%%%%%%%%%%%%%%%%%%%%%%%%%%%%%


Estas relaciones establecidas para estas furveones aasiables en una y dos dimensiones, se puede estableces aso uej para una función en tres dimensiones (3D) 0 usando otra notación $R_3$, aqui para este caso tendríamos tres rectas coordenadas mutuamente perpendiculares respectileamente hablando del eje $x$, eje $Y$ y por vitimo al eje ż, para definir un punto cuales quiera, geweralmente se utiliga la triada ( $\left.x_0, Y_0, z_0\right), 101$ quales necesariamente asumirán respectiuamen te fiora fines practicos, los walores $(a, b, c)$, el punto de intersección de los eses, de coordenadas se conoce como el origen de dicho sutrinat los ejes da coordenadas tomados en pares, definen a su uej tres planos de coordenadas, asi uamos a tenvel respecticeamente los planosixy (elex y eley) yz (ajey yejez) y xz (ejex yejez), a cada punto $P(a, b, c)$, lea a estar asociado para $x$, el cualor de a, para y,el ealor de b y para zy se deternina como se estable ee en las sigs. figuras:
%Escaneado con CamScanner


%%%%%%%%%%%%%%%%%%%%%%%%%%%%%%%%%%%%%%%%%


asi tendriamos respecticamente para ol punto

Punto $P(a, b, c)$

Figura 1.1.3
Ejemplo 1er 1. representar respecituamente en sistemas
1) Ce coordena das 30, los puntos $A(3,2,1), B(6,3 /,-2)$ $Y(-4,-6,-3)$ :

Figura1.1.4
Escaneado con CamScanner


%%%%%%%%%%%%%%%%%%%%%%%%%%%%%%%%%%%%%%%%%


costumbue usval de prentad en os culavir fermaregyimarla. devecha baprigas susperquriones posin biass ha cala cale mano derecha. Lo la ma mon dertrojiro porques sisulo empunan los de dos del ia mano dereiha mienque sures comolse pliede der enta figurati, Suea de eve de tos $z$.
figura1.1,5:
Escaneado con CamScanner


%%%%%%%%%%%%%%%%%%%%%%%%%%%%%%%%%%%%%%%%%


1.1 Ecuaciones de lo recta en el Espacio
(1) En ol plino, se utiligan para do binir la eeve cicul, - urisiticas de la misimenerimente dos atribu tos of rame.
- Uno de los puntos porlos que pasa in ( $\left.t_1, V_1\right)$ y la pendrente
- Dospuntos pes los que pasa: $\rho_1\left(x_1, y_1\right)$ y $\rho_2\left(x_2, y_2\right)$
- Pendiente y Ordenada alorrgen. ele x que seria, "a" y pu"que mide al aistaico helorigencon en el espacio de thes dinale 1
recurbente tres paramensposs uagos a ocupar de manera los numeros de pivección. Asi uamos a tener que para la recta que pasa por el punto escalar. PUede escribirse $\overrightarrow{P Q}=t Q$ donde tesuin
$$
\begin{aligned}
	\overrightarrow{P Q}=\left(x-x_1, y-y_1, z-z_1\right) & =\langle a t, b t, c t> \\
	& =t v
\end{aligned}
$$

Igualando estos componenter
$$
x-x_1=a t \quad y-y_1=b t \quad z-z_1=c t
$$

Asi wamas a tener
Teorema 1.1 kuaclonertpara es parcios de una rerta enel por el puntb P( $\left.x_1, y_1, z_1\right)$ se presenta que diante las ecuaciones paramétricas':
$$
x=x_1+a t, \quad y=y_1+b t, z=z_1+c t
$$
Escaneado con CamScanner


%%%%%%%%%%%%%%%%%%%%%%%%%%%%%%%%%%%%%%%%%


X $x$ retricas de la recta:
$$
\frac{x-x_1}{a}=\frac{y-y_1}{b}=\frac{z-z_1}{c}
$$
ecuafiones
simétricas Ejemplo 1.11 Hallar las ecuaciones paramétricas
-cas simétn-
paralela icta ct que pal uector $v=\langle 4,-1 / 2\rangle$ por punto $(3,4,-1)$ y es
solución para hallar paralela al leector $v=\langle 4,-1 / 2\rangle$
solución para y los numeros de diección $a=4, b=-1$
figuralit. 6 asileamos a tener:
as
$x=3+4 t, y=4-t, z=-1+2 t \square$
I(Ecuaciones paramétricas) Como a, b y c son tgdos.
distintos ble 0 , esclecir: $a \neq b \neq c \neq 0$ Uamos a poder decinir un
conjunfo de ecuaciones simetricas
$$
\frac{x-3}{4}=\frac{y-4}{-1}=\frac{z+1}{2}
$$
$x($ evaciones simétricas $)$ Es lmportante agregar quedade a que es uha darlable paramétrica, puede rainib O la ecuacion patrame trica.
Escaneado con CamScanner


%%%%%%%%%%%%%%%%%%%%%%%%%%%%%%%%%%%%%%%%%


Al cambiar el ualor de t los puntos que conforman
la rectai, seluan generando, de lal manera que al
toner tener un número infinito de ualores de ti, lamos poder generar un número infinito de puntos, Por elemplo si hacemos el calor de $t=1$, wamos a tener:
$$
x=3+4(1)=7 \quad y=4-1=3 \quad y \cdot z=-1+2=1
$$

Asi el punto generado wa aser $(7,3,1)$
y las ecvaciones paramétricas serian áhora:
$$
x=7+4 t, y=3+t, z=1+2 t
$$
si el cealor de $t=2$, lvego tendríamosi
$$
x=7+4(2)=15, y=3+2=5, z=1+4=5
$$
(2)
A si elpunto generado es el $(15,5,5)$ Eemplo 1.12: Hallar un confunto de ecuaciones paramétaras para la recta que pasa por los puntos $(2,3,1)$ y $(5,-2,-3)$
Solución: se empieza por usar los punlos: $P(2,3,1)$ y $Q(5,-2,-3)$ Para hallar un leactor declrección paralarecta que pasa por $P Q, e l$ uval está dado por:
$$
20=\overrightarrow{P Q}=\langle(5-2),(-2-3),(-3-1)\rangle=\langle 3,-5,-4\rangle
$$
$\langle a, b, c\rangle=\langle 3,-5,-4\rangle$ es: $a=3, b=-5, c=-4$ usando los números de dirección y el punto $p(2,3,1)$, se obtienen las ecuaciones parametrica $x=2+3 t \quad y=3-5 t \quad z=1-4 t$
Noto: a Modida que $t$ cambio, las caraciones pirametricas siterinan los yunlos $x, y, z)(d e=$, seobtconen $P(2,3,1)$ y $Q(5,-2,-3)$
Escaneado con CarmScanner


%%%%%%%%%%%%%%%%%%%%%%%%%%%%%%%%%%%%%%%%%


1.2 Planos en el esparlo
la ecuación do un plano en el espacio se obtione a partir decun punto enelplano y unvector iormel (perpenconsidere el plano que conthone el pynto $P\left(x_1, y_1, z_1\right)$, el cual fiene unceector hormal distinto de cero $n=\langle a, b, c\rangle$, como se muestra enla figura 1,2.1. Este plano consta de tados los puntos $Q(x, y, z)$ para los que el cector PQ es ortogonal g.utilizando el producto escalar se tiene que: $n \cdot \overrightarrow{p Q}=0$

La tercera equación esta' en forma es taindar. teorema 1.2 Ecuaciónetountar
de un plano: El plario que contiene el punto $\left(x_1, y_1, z_1\right)$ y tiene uncector normal:
$n=\langle a, b, c\rangle$ Puede repusenfarse, en forma estan dar me diante laecuación.
$x\left(r_1^{-1}\right)+c\left(z-z_1\right)=0$
Reagrupando, términos, se ob tiene la forma general
de ecuacion de unplano en el espacio:
$$
a x+b y+c z+d=0
$$

Recta perpendiculara un plang: $A x+B y+C z+D=0$
se ha de verificar que dichas componentes sean
Escaneado con CarnScanner


%%%%%%%%%%%%%%%%%%%%%%%%%%%%%%%%%%%%%%%%%


2 a los coeficienter de $x, y, z$ de la ecuación del plano. Siempre que $a, b, c, A, B$ y $C$ sean todos distintos dé cero y $\frac{a}{A}=\frac{b}{B}=\frac{C}{C}$, la recta yel plano son perpendiculares
Planos paralelos y perpendiculares.
Dos planos, $A_1 x+B_1 y+C_1 z+D_1=0 \quad$ y
$$
A_2 x+B_2 y+C_2 z+D_2=0
$$
son paralelos silos coeficientes de $x, y, z$ en sus ecuadiones son proporcionales, es decirsiseveritica
$$
\frac{A_1}{A_2}=\frac{B_1}{B_2}=\frac{C_1}{C_2}
$$
(d) Dos planos y $A_1 x+B_1 y+C_1 z+D_1=0$
$$
\text { y } \quad A_2 x+B_2 y+C_2 z+D_2=0
$$
son perpendiculares, cuando se ceerifica la relación entre coeficientes:
$$
A_1 A_2+B_1 B_2+C_1 C_2=0
$$

Ejemplo 1.2in Encuentrala ecuación general del $p_{(1,-6,5) 9}$. contione los puntos $(2,5,2,13,-1,-1)$ y solución 1 para eimplear el teorema 12. Senesesitaunpunto en el plano yun vecto que sea normal al plane, se tienentros $(3,-1,-1)$ y el $(1,-6,5)$ para el pue ctor $P$, ommos atener:
$$
P(3-2,-1-5,-1-2) ; P(1,-6,-3)
$$
(1).

Parael veftor $Q$, vamos a tener $Q(1-2,-6-5,5-2)$ Eleectar normal, waa ses:
Escaneado con CamScanner


%%%%%%%%%%%%%%%%%%%%%%%%%%%%%%%%%%%%%%%%%


$$
n=P \times Q=\left|\begin{array}{ccc}
	i & j & k \\
	7 & -6 & -3 \\
	-1 & -11 & 3 \\
	i & j & k \\
	1 & -6 & -3
\end{array}\right|=(-18-33) i+(3-3) j+(-11-6) k Q
$$

De aquilas componentes $\langle a, b, c\rangle$ de esto vector serín $a=-51 i \quad b=0 \quad c=-17 k$, sustituyendo en la ecuación del plano: $-51(x-2)-17(z-2)=0$ ó: sea $-51 x+102-17 z+34=0 \Rightarrow-51 x-17 z+136=0$ 6 multiplicando la ecuación por (-1) $51 x+17 z-136=0$

Para elpuntd
$(2,5,2)$
haciendo la comprobación en $(2,5,2): 51(2)+17(2)-136=0$ comprobando en el punto $(3,-1,-1): 51(3)-17(1)-136=0$ haciendo la sustitución $-n(1,-6,5): 5(1)+17(5)-136=0$ Figuro 1.2.2

En el espacio de 3
dimensiones dimensiones dos
planos distanfes planos distahfes, se in ersactan en unarectasise intersogan se sevede
hatlar elanbulo hallar el angulo $\left(0 \leq \theta \leq \frac{\pi}{2}\right)$ formado entre ellos a partur del angulo
Escaneado con CamScanner


%%%%%%%%%%%%%%%%%%%%%%%%%%%%%%%%%%%%%%%%%


$\theta$
¿demplo 1.2.2. Hallar et punto de intersección de los planos $\quad x+2 y-z=6$
$$
\begin{aligned}
	& 2 x-y+3 z=13 \\
	& 3 x-2 y+3 z=-16
\end{aligned}
$$
tenamos tres ecuaciones lineales. la solución de estesistema nos da las, coordenadas del punto de interseccion de los 3 planos, Dicho punto es $(-1,2,-3)$
ción
(2) Eona Normalila forma normal de la ecuación de un
$$
\frac{A x+B y+C Z+D}{ \pm \sqrt{A^2+B^2+C^2}}=0
$$

Endgnde el signo del radical se conisidera opueste I de D, paraqquela distancia of sea siempte positucea. Ecuacion del plapo en unción des los segmentos que intercepta en los ejes
tae equación del plano que corta a lof ejes $x, y z$ por i puntos $d, b, c$ espectivamente, leienedada
$$
\frac{x}{a}+\frac{y}{b}+\frac{z}{c}=1
$$
Escaneado con CarnScanner


%%%%%%%%%%%%%%%%%%%%%%%%%%%%%%%%%%%%%%%%%


Distancia de un punto a un plasio.
O la distancia del punto $\left(x_1, y_1, z_1\right)$-al plano
$A x+B y+C z+D=0$ es $d=\left|\frac{A x_1+B y_1+C z_1+D}{\sqrt{A^2+B^2+C^2}}\right|$
Ángulo dè dosplanos,
El ángulo agudo $\theta$ que forman dos planos, $A_1 x+B_1 y+C_1 z+D_1=0$ y el plano $A_2 x+B_2 y+C_2 z+D_2=0$ uiene defini do pon $\cos \theta=\left|\frac{A_1 A_2+B_1 B_2+C_1 C_2}{\sqrt{A_1^2+B_1^2+C_1^2} \sqrt{A_2^2+B_2^2+C_2^2}}\right|$
Casos particulares de planosi
$A x+B y+D=0$
$\begin{array}{ll}B y+C z+D=0 & \text { Representan planos perpendicolases, }\end{array}$ $A x+C z+D=0 \quad$ Yespectieamente a les planos $x y$,
los planos $A x+D=0, B y+D=0, C z+D=0$ representan planos, respecticaimente, perpendiculaes a los ejes $x, y$ YZ.
1.2.3 faceán la ecuación dit planio queu pasa por el de componentes $7,2,-3$ i.
C
Solvción a plicando la couació́ del plano
Escaneado con CamScanner


%%%%%%%%%%%%%%%%%%%%%%%%%%%%%%%%%%%%%%%%%


2 enila farma, $A\left(x-x_0\right)+B\left(y-y_0\right)+C\left(z-z_0\right)=0$ y la condición de que los coeficientes sean proporcionales a las componentes dadas entonces: $7(x-4)+2(y+2)-3(z-1)=0$ 4 bien $7 x+2 y-3 z-21=0$
1,2,4 Hallar la ecuación del plano perpendicular, en puntos medio del segmento perpendicular, en las compónentes del $(9,4,3)$ obien ponentes del segmevio son $12,2,2$ - tiene de coordehadas $(3,3,2$ del seg mento del plano es:
a bien
$$
\begin{aligned}
	& 6(x-3)+(y-3)+(z-2)=0 \\
	& 6 x+y+z-23=0
\end{aligned}
$$
1.2.5 Hallar la ecuación def plano que pasa por el
punto (1,0-2) y. es peypendicu ara punto $(1,0-2)$ y es peypendicular a los planes $2 x+y-z=2$ y $x-y-z=3$
La familia de planos que pasan por el punto $(1,0,-2)$ es; $A(x-1)+B(y-0)+C(z+2)=0$ Paraque ung de estos planos es perpendicular
a los dios dados:
$$
\begin{aligned}
	2 A+B-C & =0 \\
	A-B-B & =0
\end{aligned}
$$

Resolviendo el sistema, $A=-2 B$ y $C=-3 B, 1 a$ ecuación pedida es:
$-2 B(x-1)+B(y-0)-3 B(z+2)=0$
- bieni $2 x-y+3 z+4=0$
Escaneado con CamScanner


%%%%%%%%%%%%%%%%%%%%%%%%%%%%%%%%%%%%%%%%%


1.2.6-Hallar la ecuación del plano que pasa por los puntos $(1,1,-1),(-2,-2,2),(1,-1,2)$ sustituyendo las cooridenadas de estos puntos en a ecuación $A x+B y+C Z+D=0$ se obtiene el sistema. $A+B-C+D=0$
$$
\begin{aligned}
	-2 A-2 B+2 C+D & =0 \\
	A-B+2 C+D & =0
\end{aligned}
$$

Despejando $A, B, C$ y $D$ resultan $D=0, A=-\frac{C}{2}$,
$B=\frac{3 C}{2}, C=C$
Sustituyendo estos ualores y diaidiendo por c Nesulta la ecuación: $x-3 y-2 z=0$
$\begin{aligned} & \text { 1.2.7.- } \text { Hallar ladistancla del punto }(-2,2,3) \text { al plans } \\ & \text { de ecuación } 8 x-4 y-z-8=0\end{aligned}$
solución: la ecuación en forma normal es:
$$
\frac{8 x-4 y-z-8}{\sqrt{64+16+1}}=\frac{8 x-4 y-z-8}{9}=0
$$

Sustituyendo las coordenadas, $d=\frac{8(-2)-4(2)-1(3)-8}{9}=$ -
clel punto.
E1 signo negatiep indicn elputo
El signo ne gatiop indica el pupto $=\frac{-35}{9}$
vel origen eston al mismo lado
1,2.8. Hallar el menor án gulo formado por lesplanes
(1) $3 x+2 y-5 z-4=0$
(2) $2 x-3 y+5 z-8=0$

Los cosenos directores de los normales a los dos planos soni
$\cos \alpha_1=\frac{3}{\sqrt{38}}$
$\cos B$
$\frac{2}{38} \cos \theta^1=5$
Escaneado con CamScanner


%%%%%%%%%%%%%%%%%%%%%%%%%%%%%%%%%%%%%%%%%


-)
$\cos \alpha_2=\frac{2}{\sqrt{38}}, \cos \beta_2=-\frac{3}{\sqrt{38}}$
$\cos \gamma_2=\frac{5}{\sqrt{38}}$
Sea $\theta$ el án gulo formado por las dos
hormales. En fonces:

Problemas propuestos: 1.2.1
1.2.1.- La forma punto-normal de la ecvacićn del plano que pasa por $(2,3,5)$ y es perpendicutar as
$$
\text { Sol: } 4 x+2 y+5 z-39=0
$$
1.2.2. - Un eector normal al plano $6 x-y+5 z-12=0$ es Sol: $(6,-1,5)$
1.2.3.- Un wector normalal plano planos $x+2 y-z=6$ y $x-3 y+4 z=3$ por los $y$ sol: $\theta=10.8933^{\circ}$
1.2.27. - de mostras guelar ectas yoerisltandelalilorsecmos (2) $2 y+z-5{ }^2 x-2 y^{\prime}+5 z=8=0 x-3 y+6-0 y^{\prime}$
Escaneado con CamScanner
$$
\begin{aligned}
	& \operatorname{cosen} \theta=\left\lvert\, \frac{3}{\sqrt{38}} \cdot \frac{2}{\sqrt{38}}-\frac{2}{\sqrt{38}} \cdot \frac{3}{\sqrt{38}}-\frac{5}{\sqrt{38}}=\frac{5}{\sqrt{38} \mid}\right. \\
	& =\frac{25}{38}=0.6578 \\
	& \theta=48^{\circ} 86 \quad 0^{\prime} 48^{\circ} 51.6^{\prime}
\end{aligned}
$$


%%%%%%%%%%%%%%%%%%%%%%%%%%%%%%%%%%%%%%%%%


1.25. - Hallarel ángulo formado por la recta $x+2 y-z+3=0,2 x-y+3 z+5=0$ y el plano $3 x-4 y+2 z-8=0$
sol: $5.54,04^{\circ}$
1.26- Hallar la ecuación entorma continua de la recta.intersección de los planos:
$$
2 x-3 y+3 z-4=0 \text { y } x+2 y-z+3
$$

Sols la intersecciónestá definida por el
$(0,-5 / 2,-1 / 3)$ y trewe como componientes;
$$
(3,-5,-7)
$$
1,2.7,-Escribir en forma pargmétrica, las, ecyaciones de la recta de interseccioun de los planos $3 x+3 y-4 z-3=0$ y
$$
x+6 y+2 z-6=0
$$

Sol: $x=6 t, y=\frac{1}{3}-2 t, z=2+3 t$
1.28- - tallar el ángulo agudo, forma do por las rectas:

Sol: $49^{\circ} \cdot 265$
$$
\begin{aligned}
	& \frac{x-1}{6}=\frac{y+2}{-3}=\frac{z-4}{6} \\
	& \frac{x+2}{3}=\frac{y-3}{6}=\frac{z+4}{-2}
\end{aligned}
$$
22.9. - Haflar el angulo agudo que forma la vecfa que qup pasapor los punta $(3,4,2),(2,3,-1)$ con la que vole $(1,-2,3)$, $(-2,-3,1)$ Sol: $36^{\circ} 19^{\prime}$
Escaneado con CamScanner


%%%%%%%%%%%%%%%%%%%%%%%%%%%%%%%%%%%%%%%%%


112N 0. - Hallarilasecuagiones de la recto, que paso por
$1.2 .12-$ Encueptre una ecuación del plano que pasa por
el punifop y tiene el ceector h como normal por
a) $P(2,6,1)$; $n=\langle 1,4,2\rangle$ soli $x+4 y+2 z=28$
b) $P(1,0,0) ; n=\langle 0,0,1\rangle \quad z=0$.
1.2.13- Determine silos planos son paralelos, perpendi-
- culares 0 ni ono, $n$ via
a) $28-b$.
a) $\begin{aligned} 2 x-8 y-6 z-2 & =0 \\ -x+4 y+3 z-5 & =0\end{aligned}$
b.)
c)
$$
\begin{aligned}
	& 3 x-2 y+z=1 \\
	& 4 x+5 y-2 z=4
\end{aligned}
$$
$$
\begin{aligned}
	& x-x+3 z-2=0 \\
	& 2 x+z=1
\end{aligned}
$$
b) Perpendiculases i) Nilo lootro
1.2.14.- Determine sila recta y el plano son paraleles
perpendiculares o nilo uno, ni lo ofro.
a) $x=4+2 t \quad y=-t \quad z=-1-4 t$ sols: a) Paralelos
$$
3 x+2 y+z-7=0
$$
b) $x=t, y=2 t_1 z=3 t$
b) Nilouno, nilodtio
$$
\begin{aligned}
	& x-4+2 z=5 \\
	& x=-1+2 t, y=4
\end{aligned}
$$
c) $x=-1+2 t, y=4+t, z=1-t$
c) Perpendiculates
$$
4 x+2 y-2 z=7
$$
1.2.15 - Encuentre una eduación del piario que satistare
- a) El plano que pasa por el origin y
que e. patalelo al plano $4 x-2 y+7 z+12=0$.
b) El plano qu ue pasa por el purito $(-1,4,2)$.
- yque contene la recto de. interseceín.
solvalones; a) $4 x-2 y+z z=0 \cdot$ b) $4 x-13 y+21 z=14$
Escaneado con CamScanner


%%%%%%%%%%%%%%%%%%%%%%%%%%%%%%%%%%%%%%%%%


superficies en el espano
En la primera parto de nuestro curso, émpezamos con la clasificacion dec waliasisuperfidiesimportantes en tress dimensionesique fueron los playos, losiciales habiamos corsto, tenilala ecuación:
$$
A x+B y+C z+D=0
$$

El sequndo tipg básico do superficies es la superficie esterica utitigaldaya sea en sucalidad de hemisferio, medio the misferio o kacuarta parte del hemisferio, lo que uendria siendo un
octan te, como podemos beven lasiguiente octante, como podemos werenlasiguiente figuras

Eigura 1.3 .1 $\left(c(x-h)^2+(y-1 k)^2+(z\right.$ Corolario: bs rerfifio escinca suyt contho es esto es uendrin nendo solo un caso particular la ecuacion (22 de ta ecuacion a, sientoesta. conocidar como lima ordiaaring te la ecuación de la es,fera, s, satrellamos us ar ecilaciomy
Escaneado con CarnScanner


%%%%%%%%%%%%%%%%%%%%%%%%%%%%%%%%%%%%%%%%%


山
de denamas los tormenos, obtenomes una ecuacions
$$
x^2+y^2+z^2+4 x+H y+I z+k=0
$$
la evación (3) es la llamada forma general de la ecuación de la esfiras con tonne 4 zanstante arbitrarias inde pendientesi por lo tan to, una superflele ésfética queda perfectamente deter. Asi por ejemplo, cuatro contons indendientes, cle terminanuna cuatro pun ous no caplaraies Gemplo 1,3.1.
tiene com: Halto la ecuación del descera,que $r=5$ i solveiónisustituyerío en 1 a ecua yéndi. ordinam delolis estery (fómuláit)
$$
(x-2)^2+(y-1)^2+(z+3)^2=5^2
$$
Siqueremas llegara la eración gezueral,
basta condesilrillar esta expresiony sun.
-pificarla \begin{tabular}{c}
	$-p i f i c a r l a$ \\
	$x^2-4 x+4$ \\
	\hline
\end{tabular}
$$
\begin{aligned}
	& x^2-4 x+4+y^2-2 y+1+z^2+6 z+9=25 \\
	& x^2+y^2+z^2-4 x-2 y+6 z+4+1+9-25=0
\end{aligned}
$$
$x^2+y^2+z^2-4 x-2 y+6 z-11=0$ Ponién dola eñ forma Ejemplo 1,3.2 Hallar la ervadig de ie estio a que. ynio ale los difimetrestalefundo por los puntos $(6,2,-5)$ y $(-2,2,9)$, solución:
Primeramente zeamos adejinkelacentio deesta esfera como el punth edio del seamento damio por dichos puntos: $\quad x_m=\frac{6-2}{2} \quad+\frac{1}{2}=y_m=\frac{1.12}{2}, \frac{4}{2}=?$ $z m=\frac{-5+9}{2}=\frac{4}{2}=2$, por 10 tunto el centionsel
Escaneado con CarnScanner


%%%%%%%%%%%%%%%%%%%%%%%%%%%%%%%%%%%%%%%%%


\&
punto $(1,1,1)$, de la misma cormayelradio de ejta esfera lo leimas a poder dealure, a paitir de la fórmula de la distancia
$$
r=d=\sqrt{(6-2)^2+(2-2)^2+(-5-2)^2}=\sqrt{4^2+a^2+(-7)^2}=\sqrt{16+0+49}
$$
$r=\sqrt{65}$, sustituyendo en la ecuacion andinama de la esfera: $(x-2)^2+(y-2)^2+(z-2)^2=65$
Desarpollendo los cuedrados, llegaremos a la ecuación general $x^2-4 x+4+y^2-4 y+4+z^2-4 z+4=65$ Luego la ee gerteral sería:
$$
x^2+y^2+z^2-4 x-4 y-4 z-53=0
$$

Eemplo 1,3.3. Halla, la equación de la essera que entie es tos puntos, radio $=$ r $=d=\sqrt{(3-2)^2+(1-2)^2+(4-3)^2}$
$$
\begin{aligned}
	& (x-3)^2+(y-1)^2+(z-4)^2=3 \Rightarrow x^2-6 x+9+y^2-2 y+1+z^2-8 y+16=3 \\
	& x^2+y^2+z^2-6 x-2 y-8 y+19+1+16-3=0 \\
	& x^2+y^2+z^2-6 x-2 y-8 y+23=0
\end{aligned}
$$

Ejemplol.3.4 Hallar la ecwación dela estera con Eentro en el purto $-5,3,4)$ y tangente al eje $x$. Solwerom sies tamigantle at eje $x$. $r=5,1$ vego sostituyendo enlin erinctón ordinama. $(x+5)^2+(y-3)^2$
$$
\begin{aligned}
	& x^2+10 x+25+y^2-6 y+9+z^2-6 z \\
	& -x^2+y^2+z^2+10 x+6 y-8 z+25=
\end{aligned}
$$
Escaneado con CamScanner


%%%%%%%%%%%%%%%%%%%%%%%%%%%%%%%%%%%%%%%%%


$+4$
Eiemplo, 13.5. Halear ly eruarión de laescera applanta in el punto ervarion de lacspere solveión: $2 x-y+2 z+5=0$
Ceamos a para hallar el radio r de la espega, $(2,2,1)$ al utiluar la distancia del punto
$$
\begin{aligned}
	& \left.d=\left|\frac{2(2)-2+2(\sqrt{2})+5 \mid}{\sqrt{2^2+(1)^2+2^2}}=\right| \frac{4-2+2+5}{\sqrt{4+1+4}} \right\rvert\,=\frac{9}{\sqrt{7}} \\
	& d=3
\end{aligned}
$$
$$
d=3 \text { duega el radio dela esfera esde } 3
$$

Luego la esfera ua a tene por eevacrór:
$$
\begin{aligned}
	& (x-2)^2+(y-2)^2+(z-1)^2=3 \\
	& x^2-4 x+4+y^2+4 y+4+z^2+2 z+1=9
\end{aligned}
$$

Luego. $x^2+y^2+z^2-4 x+4 y+2 z=0$
Euimplo 1.3.6 hallai la equarión de la esuera Vue pasa por los puntos $(1,2,1),(2,2,1),(1,1,3)$ Y $(3,3,7)$ solveión; come dertación amexaldols esfera es: $x^2+y^2+z^2+9 x+4+y+1 z+k=0$
susthivendo ferpecturimente esta raiscrón gemerat,
$(1,2, T) \Rightarrow 1^2 x^2+1^2+\lambda^2+1 \lambda^2 T+\infty=0$.
$(2,2,1) \Rightarrow 2^2+2{ }^2+2^2+211-x+k=0$
$(3,3,1) \Rightarrow 3^2+3^2+j^2+$
$$
k=0
$$

Rearreglando las ecvaciond pdin mupermat ì sistemach eruariones lineales:
Restando
$$
\begin{aligned}
	& \text { ace } \\
	& 5=3 \text { de } \\
	& 4+3+K= \\
	& 3 H+ \pm+K=
\end{aligned}
$$
Escaneado con CamScanner
$$
\begin{aligned}
	& 2(G+2 H+2+K=+6 \cdots \because(1)
\end{aligned}
$$
$$
\begin{aligned}
	& 34+3 \mu+\frac{1}{1}+\mathrm{K}=\because \quad \because \therefore(3)
\end{aligned}
$$


%%%%%%%%%%%%%%%%%%%%%%%%%%%%%%%%%%%%%%%%%


Restando la ec. 4 de la 3
$$
\begin{aligned}
	& 3 L+K=1 \quad \cdots(3) I=\frac{-10}{2}=-5 \\
	& I+K=11 \quad \cdot(4)
\end{aligned}
$$

Justitupendo $I=-5$ en la ec. (4)
$-5+K=11 \Rightarrow K=16$ Juego $G=-3 H=-7 I=-5 \quad K=16$ Luego la ecuación de la esfera, sería.
$$
x^2+y^2+z^2-3 x-7 y-5 z+16=0
$$

Ejemplo 1.3.7: Hallar las coordenadas del cento yel radio de la esfera:
$$
x^2+y^2+z^2-8 x+6 y-12 z=20
$$

Para, hallar las coordenadas del centro, debemos factoriaar los tormina coniespendienties a cado
$$
\begin{aligned}
	& \left(x^2-8 x+16\right)+\left(y^2+6 y+17\right)+\left(z^2-12 z+36\right)=20+16+9+36 \\
	& (x-4)^2+(y+3)^2+(z-6)^2=81
\end{aligned}
$$
tsto es una elposecca cimotra en el punto $(1,-3,6)$ y con radio iyuara a
Ejemplo 1.3.8: Hallar el lugar geométrico de los puntos cuyas distancias a los pontes fllos $(-3,3,-3)$ y $(4,-4,4)$ estain en relacioh $3: 4$
$$
\frac{\sqrt{(x+3)^2+(y-3)^2+(3+3)^2}}{\sqrt{(x-4)^2+(y+4)^2+(3-4)^2}}=
$$
Escaneado con CamScanner


%%%%%%%%%%%%%%%%%%%%%%%%%%%%%%%%%%%%%%%%%


2d
$$
\begin{aligned}
	& 4\left(\sqrt{(x+3)^2+(y-3)^2+(z+3)^2}=3 \sqrt{(x-4)^2+(y+4)^2+(z-4)^2}\right. \\
	& 16\left((x+3)^2+(y-3)^2+(z+3)^2\right)=9\left((x-4)^2+(y+4)^2+(z-4)^2\right) \\
	& 7 x^2+168 x+7 y^2+168 y+7 z^2+168 z=0 \\
	& 7\left(x^2+24 x+144\right)+7\left(y^2+168 y+144\right)+7\left(z^2+168 z+144\right)=21(144) \\
	& x^2+24 x+144+y^2+24 y+144+z^2+24 z+144=3(144) \\
	& (x+12)^2+\left(y^2-12\right)^2+(z+12)^2=3(144) \\
	& (5 \text { to es }
\end{aligned}
$$

Es to es una esfera con centro en $(-12,12,-12)$ y con radio igudi a $12 \sqrt{3}$
1.3.1- Hallar lo equación de la. 2.- Stailfaty ${ }^2 z^2-4 x+2 y^3-6 z-2=?$
$$
2 x+2 y-z+5=0
$$
2.3.6.- Aallar la eavacion lela $196 x$ esera centto el. punto $(3,1,3)$ y es tangente ac ejey.
® $\$ 3.7$ - Obtoner la ecuni que heng filos puntos Sol: $\left(x-\frac{5}{2}\right)^2+(y-1)$
Escaneado con CarmScanner


%%%%%%%%%%%%%%%%%%%%%%%%%%%%%%%%%%%%%%%%%


superficses cilińdricas.
un tercertipo ale supexficie en el espacio quese puederapteciar en a aiqurali3,2 Podemos considerar que este alingro setialia generado par ina secia ueertical que se mueue alo largo de la cigcunferemedad $x^2+y=a^2$ del plano $x y$ Podemos llamarle a esta cirunterencia curcea generatri del cllindro, dada la uelefinición
Aguka 1,3,2
.
Escaneado con CamScanner


%%%%%%%%%%%%%%%%%%%%%%%%%%%%%%%%%%%%%%%%%


tanto todos sus nuntos situpdas a una distancia. sila de una inna reta, eleje deynaindro.El solido encerrado por estasuperficie ypor dos pla nos perpendiculares alepe tambrén es lomo cimado cilindro. Este sollido es etilizado como una superficil Gouslana.
este cilindro heve como ecvaciónes: $x^2+y^2=a^2$ las hectasgeneratricesson para-

Figura 2,1.
Escaneado con CamScanner


%%%%%%%%%%%%%%%%%%%%%%%%%%%%%%%%%%%%%%%%%



$d B$
cilindiricas:
2.- Cilindro elípticoi tomando como directrin
unaelipse, se puede generqu unal superie,
cilindira eliptical que indure a os ciindie circularica eliptica (que incluye a los cilindros circulares cuando los semieres de la elipse
soniguales) 3.- cilindros oblícuas:
a) Do base elphicuas,' angulo iectoy la superficie laferal esuna suppritlele. cilindricade reyoluctiós,i9 seccions se cta (perpendicius) aleje es.un cifuloy las
B) bases son elipses.
cilindro oblicuos debase circulate elangulo entreel eje y las base no e\} un
1].
angulo recto. Ia sección recta o.herpendicular al ejees cha elipse y las bases, son


%%%%%%%%%%%%%%%%%%%%%%%%%%%%%%%%%%%%%%%%%


ఎ)

El tercer tipo de superficie del espacio es ronorndy como supericie ilindica, óabrocoridamegifo omo elindro.
(4)

Figura 1,3,2
Para el cilindro $x^2+y^2=a^2$ podemos considenar como Le sidcunseroncia $x^2+y^2 y^2$ comolo duecter? de superfiele ilindsica y, in? diectel?
Deffipielón de superficle cifin atricat ts unasuperficie cilindricq la,generada por unareck quese muecee de tal manera que se mantrerze slempre parale la a una hecta diga dada y pasa slempere
(2)
por una curcia lita dade
Escaneado con CamScanner


%%%%%%%%%%%%%%%%%%%%%%%%%%%%%%%%%%%%%%%%%


los plamis coordenzados. Poreiemploiseac una porciondela drectril conteridatenel plano y $z$, y sean $[\alpha, 7 \beta, \gamma]$ los Corenos directores de la genératro de la superficie cilindrica, podemos escribit las ecuce

Figura:1.3.3.
Sea $P(x, y, z)$ un punlo rualquera de la superficie, I supgngamos que la generatrizue pafa par corta a cenel pufto pro, y'z'. Ento
ecuaciones de estageneratre son:
$$
\frac{x}{a}=\frac{y-y^{\prime}}{b}=\frac{z-z^{\prime}}{c}
$$

Además, comp $P^{\prime}$ 'esta sobre $C$ sy's coordenacks sa- tisfacer a (2) y luego $f\left(y^{\prime}, z^{\prime}\right), x^{\prime}=0$

Eigmio i3.9Hallar la quuación de la, superficie
$$
y^2=4 x, z=0
$$
contenida en el planoxy y cuyas generatioes tienen por números directores $[1,1,3]$
Escaneado con CamScanner


%%%%%%%%%%%%%%%%%%%%%%%%%%%%%%%%%%%%%%%%%


supferficiecor ta ta la directrig en el punto $p(x ; y, 0)$ entonges las equaciones
$$
\frac{x-x^{\prime}}{1}=\frac{y-y^{\prime}}{1}=\frac{z}{3}
$$
también como p'éstá sobre la parabola, fenemosi
$$
y^{\prime 2}=4 x^{\prime}, z^{\prime}=0
$$

Eliminaindo $x^{\prime} y^{\prime} z^{\prime} z^{\prime}$ delas ecuaeiqnes (2) y'(3) por sustityeión de cealores dex' y y'dados por las ecuaciones, ob tenemos
$$
9 y^2+z^2-6 y z-36 x+12 z=0(4)
$$
que esla ecuación buscada.
Y Figura.1.3.4
alsuperficie(4) sobre el planbeytlesiela traza dectis (i) superficie(4) :
reorema th el espacio, la gráfira de vina ecuacrón en dos do las tres cuivíubles x, Yi y z es an cilindr. cuyas generatrífs son paralelas al eje dela reariable queifabta.
Elemplol.3.10 Dioujar las superficies epeperen-tadas por las ecuaciones siguientes
a) $z=x^2$
b) $z=\operatorname{sen} y \quad 0 \leq y \leq 2 \pi$
solución:
a) La grá́ica de un cilindrocuya gurga generatrin, al elex como se muestra en lafigura 1.3.5.
Escaneado con CarmScanner


%%%%%%%%%%%%%%%%%%%%%%%%%%%%%%%%%%%%%%%%%


Figura. 13,5 son pasat, al al a $\times$ romo se aprecra en/a ¡igerat,3,6:
Escaneado con CamScanner


%%%%%%%%%%%%%%%%%%%%%%%%%%%%%%%%%%%%%%%%%


casa tir lioresab weth /hm*"
Elemplo, lustratuo 1.3.11 Mos trar la grafica de $x^2=12$ yendeos suya divectrices son a) parabola $x=12$ en el plano $x y$ con regladura ogenergtria para$3 x^2+4 y^2=12$ b) Ee cilin dro to ya dire ir es alila ale c) Et cilindro en el plano xycon ueglactura,p an ipárbola ejez. $16 x^2-9 . y^2=14 \frac{q u e}{}$ en elplano $x y$ y generáring parallola al ele
a) El cilindro cuya direcifing es la pará $60^{\prime} / x^2=12 y$ si solucion investra en la figura 1.3.7:

Figura 1.3 .7
b) $\epsilon$ critindrocuna direc $\frac{1}{4}+\frac{y^2}{3}=12$ se muestra.en
r) La figurali, lision muestra cilindro. la figura $1.3,8$ directrid es ta hipóndoda $16 x^2-4 y^2=$ $0 \cdot \frac{x}{9}-\frac{y}{16}=1$


%%%%%%%%%%%%%%%%%%%%%%%%%%%%%%%%%%%%%%%%%



Figura 0.3 .9)
)
Nota: para la supericicie cilindrica detinicla enel inciso a) ramos a tener un dilindto parabólico inciso b) el cilindro ua a sos eliptico yaque us dirigidopor una elipse y para el cilitadio del inciso ci, uamos a decir que se trata, de uncilindro hiperbólico, ya que una hiperbola es la gue actúa como directrild del lugar geome trico,
Definición: si una curua plana se hacg girar aliededor de una recta fija que estaen plano della e perag, la superfleie generada sellama su perficie de teucollós la recta fija se llama eje de re ceolución de ta superpere y la curcea plana se llama curcea generatris) en woluente.
Escaneado con CamScanner


%%%%%%%%%%%%%%%%%%%%%%%%%%%%%%%%%%%%%%%%%


Problemas Propuestos 1.3, Soperficies cilntriat.
i.3. 8 race la secuión transecersal del allindro
a) en el plano que seendica
a) $x^2+4 y^2=16 \quad$ plano $x y$
b) $z=e^y$ planoyz
b) $z=e^y$ planoyz
li3.2 bibulu elcilindro que tenga laecuación
que se indica.
a) $9 x^2+4 y^2=16$
b) $x=|z|$
c) $z=2 y^2$
d) $y^2=x^3$
1.3.9.0b ten ga una ecuación de la quper cicie
2 de recolucon duba se eeceide al jeirar la Curcea plana dada
trace la suerficie
a) $x^2=12$ y en el planoxy alrededor del gle y Soli $x^2+z^2=12 y$
b) $x^2+4 z^2=24$ en el planox $z$ alkede dor delejex

Sol: $x^2+4 y^2+4 z^2=44$
i. 3.10 ien los sigulentes incisos obtenga una cheriaginerongaray e e e para la suserficie
a) $x^2+y^2+z_2^2=48$

Sol: eleje $x: x^2+z^2=48$
b) $x^2+y^2-z^2=12$

Sol $x^2-z^2=12$ i eje $=$
c) $x^2+z^2=|y|$

Sol: $z=\sqrt{|y|}:$ eje
USUnL
Escaneado con CamScanner


%%%%%%%%%%%%%%%%%%%%%%%%%%%%%%%%%%%%%%%%%


lonalion de cilndros elntro-
Tabla 1.3 .1
con centro: $K x^2+L y^2+M z^2=N$
)
I) Los dos nulos El plano XY
II) Uno hulo El cilindro pasataitica
II) El mismo El parabeloide etipíaco
IV) Signo distint
Ee panaboloide hitoricit=o
Escaneado con CamScanner


%%%%%%%%%%%%%%%%%%%%%%%%%%%%%%%%%%%%%%%%%


superficies cuádricas
tl cuartohpo de superticies que wamos a analizar sonlas superficiercuadricas.la ecuacióngeneral de una superficie cuádrica es una eiluación de segundo grado, como sepuedeseer a con tinusun,
$$
A x^2+B y^2+C z^2+D x y+E x z+F y z+G x+H y+I z+K=0 .
$$

En doride por lo menos unio de los ciseficientas. A, B, C. D, E Y Es difergntea a, Una superficie cuya ecuaciónje, de la tormo de (i) se conore como superficie evadrica a do forma nas quneral romo cuidrica, po demos asociar estee conceptocon los lugares gepmé tricos uestudrades con on lericmad por elempidia superficie esferica es upa cuárica, Tembién losonilas superficees cilindarica Y cóniça cuyà ecuiciones sean de segundo grado son cuadtrcas asi lamas a encontrar el cilindroy el conacuadrica, De la misma marefa cualquier su pe ficie, sjada representada por una ecuadibndicequindo gradose conore
como ruadica réla
siuna superficie cuádrisa es cortada por un
siuna superficiecuádrifa es coptada por un
plano cealquiera, la curua de cnterseccion
es una seeqion conica ouna formalimitede es una section, conica o una formalimitedes una seciron cónica. asi coamos aiteneri
a) parábola c.) (ircunferencia
d) Atpesbolla
Escaneado con CarnScanner


%%%%%%%%%%%%%%%%%%%%%%%%%%%%%%%%%%%%%%%%%


as1 como iimos que al serpheriechodo una roádrica como un conolas colin uetol obtenemor tadai las
(c)
sinembargo assque la discuson Dx+Ey+F=0 traslaciones Y rotacionis de e/es a decuadas
- po demos aneligar mas puntualmentelas ronicas, asi a amas a teneri
$$
\begin{aligned}
	& M x^2+N y^2+P z^2=R \\
	& M x^2+N y^2=5 z
\end{aligned}
$$

Para las superficter con una ecvación similara (1) pode $m$ os leer que terven. un centro de sienetra. el origen y poreso sellgman: cuadricaicon centro de simetria y sellaman portonb cuddricus sin centro.
acontinvación teamos a analija pór medio de una tabla las caracter sicas de cada uno
D
de pllas
Porfacili dad de notiación, estos ualores se Hemplazan (por $a^2=\frac{R}{M}, b^2=\frac{R}{N}$ y $c^2=\frac{R}{2}$, parn las curcens del tipo (7) y $a^2=\frac{5}{M} ; b^2=\frac{5}{N}$, siendo eneste ca 80,15 elualor pan el coeficiente correspondiente al'eje de. simetira en el lugar geometrico de terminadd:


%%%%%%%%%%%%%%%%%%%%%%%%%%%%%%%%%%%%%%%%%


Clasificarión de loscuádricas

M/Ny P: los lugares geométricos correspondientes estardn dados entoncos como para $R>0$

TiPO(II) $H x^2+M y^2=5 z$

M y Ni los lugates geométincos correspond coutu estarit dados entonces coma para $S>0$,
Escaneado con CamScanner


%%%%%%%%%%%%%%%%%%%%%%%%%%%%%%%%%%%%%%%%%


Escaneado con CamScanner


%%%%%%%%%%%%%%%%%%%%%%%%%%%%%%%%%%%%%%%%%


Escaneado con CamScanner


%%%%%%%%%%%%%%%%%%%%%%%%%%%%%%%%%%%%%%%%%


Ejemplos Resueltos "Cuádricas
W14.REemplar Pesvel tos i Analice la ecuacion : $\frac{x^2}{9}+\frac{y^2}{4}-\frac{z^2}{16}=1$
Solucyon los teazss en las ejes, se obtrenen haciendo $z=0$, Soluctope su gratica
$z=0$, Plano $x y$ respectiva:
$\vec{x}=0$ : Plano $y z ; \frac{y^2}{9}+\frac{y^2}{4}=1$, Una elipse
$y=0$; plano $x z=\frac{\frac{x^2}{9}-\frac{z}{16}}{16}=1$, un a hiperbolas obsejuese que a a sustruir $\approx \pm \pm 4$ enla ecuadión ariginal,
se obtiene, $\frac{x^2}{9}+\frac{y^2}{4}-\frac{16}{16}=1$
$$
\frac{x^2}{9}+\frac{y^2}{4}=1+\frac{16}{16} \Rightarrow \frac{x^2}{9}+\frac{x^2}{4}=2
$$
$$
b^2=8=3=2.828
$$
que es und ellipse
con ajo mayor al 4.24
x e ep.menok $b=2.8284$,
T para esta elpse terideriamosi
$$
c=\sqrt{a^2-b^2} \text { drstanc, }
$$
$$
\begin{aligned}
	& c=19-8 \times \sqrt{10} \\
	& c=3.16
\end{aligned}
$$
y uniá excontricidad
$$
e q \frac{c_2}{Q}=\frac{316}{4.24}=7.45
$$

Figura 14.4
Escanereado con Cams Samner


%%%%%%%%%%%%%%%%%%%%%%%%%%%%%%%%%%%%%%%%%


1.4.2. Estuidiar y representar la superficie:
$$
\frac{x^2}{16}+\frac{y^2}{25}+\frac{z^2}{9}=1
$$

Solvcióni estasupersicie es simétrica con respecto alorigen,
como a os pinqt coordenados. Corta a los eje $x, y$ y zen, los puntar $\pm 4, \pm 5, \pm 3$.
su traza con el plano xyes la elipse de ecuación.
$$
\frac{x^2}{16}+\frac{y^2}{25}=1 \text { y semiejes } 4 \text { y } 5
$$
asismono lostrazas con los planes $x z$ e yz son también el ipseses esta superficie es un elipsode como sepade leer en la figuta 1,45 1.43, pemostrar gue la ecuación
$$
4 x^2+3 y^2+z^2+8 x-6 y+2 z-4=0
$$
es unelipsorde, hallarsU
centro, y las Yongitudes de los
Figura $1.4,5$
senieses. solucion para resoluer esta situacibn lea mos a separar los terminos que tenganx, es detir se agrupa poruep prinara a los ferminos can it y pos-
- teriormente y y duego z, luego tendriamos:
$$
4\left(x^2+2 x \quad\right)+3\left(y^2-2 y \quad\right)+\left(z^2+2 z\right)=4
$$
completarido toinomio cuadrado perfecto.
$$
4\left(x^2+2 x+1\right)+3\left(y^2-2 y+1\right)+\left(z^2+2 z+1\right)=4+4+3+1
$$
$4(x+1)^2+3(y-1)^2+(z+1)^2=12$
mu(tiplicando por $1 / 2$ toda la ecuaclón.
$$
\frac{4(x+1)^2}{12}+\frac{3(y-1)^2}{12}+\frac{(z+1)^2}{12}=\frac{12}{12} \text {, llego } \frac{(x+1)^2}{3}+\frac{(y-1)^2}{4}+\frac{(z+1)^2}{12}=\frac{1}{1}
$$
suego sería un elipsorde (on centra en $(-1,1,-1$ ) y semiejes $a^2=3 ; a=\sqrt{3} \quad b^2=4 \quad b=2 ; c^2=12 \quad c=2 \sqrt{3}$
Escaneado con CamScanner


%%%%%%%%%%%%%%%%%%%%%%%%%%%%%%%%%%%%%%%%%


1.4.4: Demostras que el lugar geométrico de los puntos cuya suma de distancias a los puntos fijos $(3,1,5)$ y $(3,-1,5)$ es constante igual a 4 , es un elipsoide Soluciónisea la suma de las distancias:
$$
\sqrt{(x-3)^2+(y-1)^2+(z-5)^2}+\sqrt{(x-3)^2+(y+1)^7+(z-5)^2}=4
$$

Rearreglando la ecuación:
$$
\sqrt{(x-3)^2+(y+1)^2+(z-5)^2}=4-\sqrt{(x-3)^2+(y+1)^2+(z-5)^2}
$$

Eeuando cuadrado
$$
\begin{aligned}
	& (x+3)^2+(y-1)^2+(z-5)^2=16-8 \sqrt{(x-3)^2+(y+1)^2+(z-5)^2}+(y-3)^2+(y+1)^2+\left(-f-(5)^2\right. \\
	& \text { Es to es: }(y-1)^2=(y+1)^2+16-8 \sqrt{(x-3)^2+(y+1)^2+(z-5)^2} \\
	& y^2-2 y+y=y^2+2 y+1+16-8 \sqrt{(x-3)^2+(y+1)^2+(z-5)^2} \\
	& -2 y-2 y=16-8 \sqrt{(x-3)^2+(y+1)^2+(z-5)^2}
\end{aligned}
$$

Dicidiendo entre 2 y haciendo transpocisión dee términos
$$
4 \sqrt{(x-3)^2+(y+1)^2+(z-5)^2}=y+8
$$

Eleceando al cuadrado, para el iminar la raíz

Aplicando regla del Sand wich
$$
\frac{\frac{16}{1}(x-3)^2}{\frac{784}{15}}+\frac{\frac{15}{1}\left(y+\frac{8}{15}\right)^2}{184}+\frac{16(z-5)^2}{18}=1 \frac{(x-3)}{\frac{4}{15}}
$$
Escaneado con CamScanner
$$
\begin{aligned}
	& 16(x-3)^2+16\left(y^2+2 y+1\right)+16(z-5)^2=y^2+16 y+64 \\
	& \left.16(x-3)^2+16 y^2+32 y+16+16(z-5)^2=y^2+16 y+64+25\right) \\
	& 16(x-3)^2+16 y^2-y^2+32 y-16 y+16-64+16(z-5)^2=0 \text { elipsoide con } \\
	& \begin{array}{l}
		16(x-3)^2+15 y^2+16 y-48+16(z-5)^2=0 \\
		16(x-3)^2+15\left(y^2+\frac{16}{15}+\frac{64}{225}\right)+16(z-5)^2=48+\frac{64}{15}
	\end{array} \\
	& (3,-8 / 5,5 \text { ) } \\
	& 16(x-3)^2+15\left(y+\frac{8}{15}\right)^2+16(z-5)^2=\frac{784}{15} \\
	& \frac{16(x-3)^2}{\frac{784}{15}}+\frac{15\left(y+\frac{8}{15}\right)^2}{\frac{784}{15}}+\frac{16(z-5)^2}{\frac{784}{15}}=1 \\
	& a^2=3.2666 \quad a=1,807 \\
	& c^2=3.2
\end{aligned}
$$


%%%%%%%%%%%%%%%%%%%%%%%%%%%%%%%%%%%%%%%%%


contresperto a los ejes cuordenados.
solución: la ecuación de un elipsorde será
$$
\frac{x^2}{a^2}+\frac{y^2}{b^2}+\frac{z^2}{c^2}=1
$$
sustituyendo las coordenadas de cada uno de nuestros puntosi le amos a ob tener las siguientes ecuaciones sustityendo en la ecuacion ondinaria
$$
\begin{aligned}
	& \frac{9}{a^2}+\frac{9}{b^2}+\frac{36}{c^2}=1 . \\
	& \frac{0}{a^2}+\frac{0}{b^2}+\frac{81}{c^2}=1 . \\
	& \frac{9}{a^2}+\frac{36}{b^2}+\frac{9}{c^2}=1
\end{aligned}
$$
de la ecuación (2) $c^2=81 \quad c=9$
( )
$$
\begin{aligned}
	& \frac{9}{a^2}+\frac{9}{b^2}=1-\frac{36}{8.1} \Rightarrow \frac{9}{a^2}+\frac{9}{b^2}=\frac{45}{81} \Rightarrow \frac{1}{a^2}+\frac{1}{b^2}=\frac{5}{81} \\
	& \frac{9}{a^2}+\frac{36}{b^2}=1-\frac{9}{8.1} \Rightarrow \frac{9}{a^2}+\frac{36}{b^2}=\frac{72}{81} \Rightarrow \frac{1}{a^2}+\frac{4}{b^2}=\frac{8}{81}
\end{aligned}
$$
sustituyendo en (1) y en (3)
testando (1) de (2)
$$
\begin{aligned}
	\therefore \frac{1}{a^2}+\frac{4}{b^2} & =\frac{8}{81} \\
	\frac{1}{a^2}+\frac{1}{b^2} & =\frac{5}{81} \\
	0 \quad \frac{3}{b^2} & =\frac{3}{81}
\end{aligned}
$$
$b^2=81 b=9$ Luego sustitryendo
en (2)
$$
\frac{1}{a^2}=\frac{8}{81}-\frac{4}{81} \quad \frac{1}{a^2}=\frac{4}{81}
$$
$a^2=\frac{81}{4} \quad a=\frac{9}{2}$, sustiluyend
$a, b$ y $c$ en laeci ordimand
$$
\frac{x^2}{\frac{81}{4}}+\frac{y^2}{81}+\frac{z^2}{81}=1
$$
$\Rightarrow \begin{aligned} & 4 x^2+y^2+z^2=81 \text { que } \\ & \text { ecuación del lugar }\end{aligned}$
ecuación del lugar geométro.
Escaneado con CamScanner


%%%%%%%%%%%%%%%%%%%%%%%%%%%%%%%%%%%%%%%%%


1.46: Demostrar que la ecuación siguiente es un elipsoide. Hallar su centro y las longitudes de los semiejes.
$$
4 x^2+3 y^2+2 z^2-16 x+12 y-12 z-6=0
$$
solución: reagrupando, y completando los cuadra-
-dos en estalexpresión:
$$
\begin{aligned}
	& 4\left(x^2-4 x\right)+3\left(y^2+4 y\right)+2\left(z^2-6 z\right)=6 \\
	& 4\left(x^2-4 x+4\right)+3\left(y^2+4 y+4\right)+2\left(z^2-6 z+9\right)=6+18+12+16 \\
	& 4(x+2)^2+3(y+2)^2+2(z-3)^2=52
\end{aligned}
$$

Dicidiendo todo entre 52
$$
\begin{aligned}
	& \frac{4(x+2)^2}{52}+\frac{3(y+2)^2}{52}+\frac{2(z-3)^2}{52}=1 \\
	& \frac{(x+2)^2}{13}+\frac{(y+2)^2}{14}+\frac{(z-3)^2}{26}=1
\end{aligned}
$$

Que es un elip soide de centro el punto $(-2,-2,3)$ y semiejes $\sqrt{13}, \sqrt{14}$ y $\sqrt{26}$ siendo $a, b$ y $c$.
1.47 Demostrar quelel lygar geométrico de los puntos cuya suma de distancias a los puntos fijos $3,4,5$ y y 3 , 4,5 es constante e iqual al 12
esyn elipsolde. Hallarsucentroy las longitudes
de las semiejes: delas semiejes:
$$
\begin{aligned}
	& \sqrt{(x-3)^2+(y-4)^2+(z-5)^2}+\sqrt{(x-3)^2+(y+4)^2+(z-5)^2}=12 \\
	& \sqrt{(x-3)^2+(y-4)^2+(z-5)^2}=12-1(x-3)^2+(y+4)^2+(2-5)^2
\end{aligned}
$$

Eseuando al cuadrado, ambarmiembro
Escaneado con CamScanner
$$
\begin{aligned}
	& \text { de laecuacion. }\left(\sqrt{(x-3)^2+(y-4)^2+(z-5)^2}\right)^2=\left(12-\sqrt{\left.(x-3)^2+(y+4)^2+1=-5\right)^2}\right)^2 \\
	& (x,-3)^2+(y-4)^2+(z+5)^2=144-24 \sqrt{(x-3)^2+(y+4)^2+(z-5)^2}+(x / 3) \\
	& (y-4)^2=144-24 \sqrt{(x-3)^2+(y+4)}+\overline{(z-5}^2+(y+4)^2 \\
	& y^2-8 y+16=144-24 \sqrt{(x-3)}+\left(+^4\right)^2+(2-5)^2+y^2+8 y+1 /
\end{aligned}
$$
$$
\begin{aligned}
	& \left(24 \sqrt{(x-3)^2+(y+4)^2+(z-5)^2}\right)^2=(144+16 y)^2
\end{aligned}
$$


%%%%%%%%%%%%%%%%%%%%%%%%%%%%%%%%%%%%%%%%%


$$
\begin{aligned}
	& 576\left((x-3)^2+(y+4)^2+(z-5)^2\right)=20736+4608 y+256 y^2 \\
	& 9\left((x-3)^2+(y+4)^2+(z-5)^2\right)=324+72 y+4 y^2
\end{aligned}
$$

Desarrollando los cuadrados:
$$
\begin{aligned}
	& 9(x-3)^2+9 y^2+7 z y+144+9(z-5)^2=324+7 z 2 y+4 y^2 \\
	& 9(x-3)^2+5 y^2+9(z-5)^2=180
\end{aligned}
$$

Dieidiéndo entre 180
$$
\frac{(x-3)^2}{20}+\frac{y^2}{36}+\frac{(z-5)^2}{20}=1
$$
esto es un elipsoide con centro en $(3,0,5)$ y
semiejes $a=\sqrt{20}=2 \sqrt{5} \quad b=6 \quad c=2 \sqrt{5}$
1.48 Hallar ${ }^2$ ecuaciŕ' del cliese qúeque pasa for los ivifos $(6,39)(0,9,0), 66,63$ y es simetricoron respecto a ios planos soordenous.?
La eruación delelipsorde gee, cis sinétrico los ejes coordenados se define por $\frac{x^2}{a^2}+\frac{y^2}{b^2}+\frac{z^2}{c^2}=1$ sustituyendo paracada uno de las puntos:
$$
\begin{aligned}
	& \frac{36}{a^2}+\frac{36}{b^2}+\frac{81}{c^2}=1 \ldots a(1) \\
	& \frac{0}{a^2}+\frac{81}{b^2}+\frac{0}{c^2}=1 \cdots(2) \\
	& \frac{36}{a^2}+\frac{36}{b^2}+\frac{9}{c^2}=1 \ldots(3)
\end{aligned}
$$
de la ecuación (2), tenemos $b^2=81$
sustityrendo en (1) yen (2)
$$
\begin{array}{ll}
	\frac{36}{a^2}+\frac{9}{81}+\frac{81}{c^2}=1 & \frac{36}{a^2}+\frac{81}{c^2}=1-\frac{1}{9} \\
	& \frac{36}{a^2}+\frac{81}{c^2}=\frac{8}{9} \cdots\left(1^{\prime}\right) \\
	\frac{36}{a^2}+\frac{36}{b 1}+\frac{9}{c^2}=1 & \frac{36}{a^2}+\frac{3}{2}=1-\frac{4}{a} \\
	\frac{36}{a^2} \frac{1}{c^2} &
\end{array}
$$
Escaneado con CamScanner


%%%%%%%%%%%%%%%%%%%%%%%%%%%%%%%%%%%%%%%%%


Resolvicendo el sistema de ecuaciones formado por $\left(1^{\prime}\right)$ y $\left(2^{\prime}\right)$

Sustituyendo
111 Hallar la naturalezadela cúdtricacuya
1.4.9 Hallón es $4 x^2+2 y^2-3 z^2+8 x-16 y+12 z=6$
Agrupando y completando cuadra dos:
$$
\begin{aligned}
	& 4\left(x^2+2 x+1\right)+2\left(y^2-8 y+16\right)-3\left(z^2-4 z+4\right)=(1+4+32-12 \\
	& 4(x+1)^2+2(y-4)^2-3(z-2)^2=24
\end{aligned}
$$
pividiendo entre 24
$$
\left(\frac{x+1}{6}\right)^2+\frac{(y-4)}{12}-\frac{(z-2)^2}{8}=1
$$
decoordenatals $z$ a secciones pro
porpianos pa. alelos $x z$ ida
Escaneado con CamScanner
$$
\begin{aligned}
	& \frac{36}{a 1^2}+\frac{81}{c^2}=\frac{8}{9} \\
	& =\left(\frac{36}{a^2}+\frac{9}{c^2}=\frac{5}{9}\right) \\
	& c^2=\frac{9(72)}{3} \\
	& 0+\frac{7^2 2}{c^2}=\frac{3}{9} c^2=3(72) \\
	& c^2=216 \quad c=\sqrt{216} \\
	& C=3 \sqrt{24}
\end{aligned}
$$
$$
\begin{aligned}
	& \frac{36}{a^2}+\frac{81}{216}=\frac{8}{9} \Rightarrow \frac{36}{a^2}=\frac{8}{9}-\frac{81}{216} \\
	& \frac{8}{9}-\frac{9}{24}=\frac{192-81}{216} \quad \frac{36}{a^2}=\frac{8}{9}-\frac{9}{24} \\
	& \frac{36}{a^2}=\frac{111}{2 / 6} \\
	& a^2=\frac{36(216)}{111}=\frac{12(216)}{37}, \quad a=\frac{\sqrt{12(216)}}{\sqrt{37}}
\end{aligned}
$$


%%%%%%%%%%%%%%%%%%%%%%%%%%%%%%%%%%%%%%%%%


4. 10 : Hallar la naturaleza de la cua'drica de ecuación
$3 x^2-2 y^2-3 z^2-18 x+6 y-12 z-21=0$
$5\left(y^2\right.$
$$
\begin{aligned}
	& \text { Solució }: 3\left(x^2-6 x+9\right)-2\left(y^2-6 y\right)-3\left(z^2+4 z\right. \\
	& \begin{array}{l}
		3\left(x^2-6 x+9\right)-2\left(y^2-6 y+9\right)-3\left(z^2+4 z+4\right)=25+27-18-16 \\
		3(x-3)^2-2(y-3)^2-3(z+2)^2=52-34 \\
		3(x-3)^2-2(y-3)^2-3(z+2)^2=18 \\
		\left(\frac{x-3)^2}{6}-\frac{(y-3)^2}{9}-\left(\frac{z+2}{6}\right)^2=1\right.
	\end{array}
\end{aligned}
$$

Que es un hiperboloide de 2 hojas con su centro en el punto $(3,3,-2)$ y eje real paralelo al de coordenasas.x. 1.4.11: Hallar el lugar aeométrica de los puntas
c. y a diferencia de digtameias a los puntos flos
$(2,-3,1)$ y $(2,3,1)$ sea igual a 4 . $(2,-3,1)$ y $(2,3,1)$ sea igual a 4.
solución $\sqrt{(x-2)^2+(y+3)^2+(z-1)^2}-\sqrt{(x-2)^2+(y-3)^2+(z-1)^2}=4$
$$
\text { obien: } \sqrt{(x-2)^2+(y+3)^2+(z-1)^2}=4+\sqrt{(x-2)^2+(y-3)^2+(z-1)^2}
$$

Eleceando al cuadrado:
$$
\begin{aligned}
	& (x-2)^2+(y+3)^2+(z-1)^2=16+8 \sqrt{(x-2)^2+(y-3)^2+(z-1)^2}+(y-1)^2+(y-3)^2+(3-1)^2 \\
	& y^2+6 y+9=16+8 \sqrt{(x-2)^2+(y-3)^2+(z-1)^2+y^2-6 y+9} \\
	& 12 y-13=8 \sqrt{(x-2)^7+(y-3)^2+(2-1)^2}
\end{aligned}
$$

Dividiendo toda la ceuación entie 4
$$
3 y-4=2 \sqrt{(x-2)^2+(y-3)^2+(z-1)^2}
$$

Elevando al cuadrado
$$
\begin{aligned}
	& (3 y-4)^2=4(x-2)^2+4 y^2-24 y+36-4 \\
	& 9 y^2-24 y+4 y^2+3 y^2 y-4(x-2)^2-4(\bar{y}-1)^2= \\
	& 5 y^2-4(x-2)^2-4(z-1)^2=20 \\
	& \frac{y^2}{4}-\frac{(x-2)^2}{5}-\frac{(z-1)^2}{5}=
\end{aligned}
$$
๖
Es to es un hiper Golorde
el punto $(2,0,1)$ ion sem.
1.2
"es un hipetboloide d.
Escaneado con CamScanner


%%%%%%%%%%%%%%%%%%%%%%%%%%%%%%%%%%%%%%%%%


)
1.4.12. Halfar el lugar geométrico de lospuntos
cuy aystancia al ounto $(3,2,-3$ es es triple
de ta. ortes ponde cle la. iorres pond, ente ae eje 4 :
Solución: $\sqrt{(x-3)^2+(y-2)^2+(z+3)^2}=2 \sqrt{x^2+z^2}$
Elenarido al cuadŗado, reduciendo y simplificando
$$
\begin{aligned}
	& (x-3)^2+(y-2)^2+(z+1)^2=4 x^2+4 z^2 \\
	& x^2-6 x+9-4 x^2+y^2-4 y+4+z^2+6 z+9-4 z^2=0 \\
	& \left(y^2-4 y+4\right)-3\left(x^2+2 x+1\right)-3\left(z^2-2 z+1\right)=-9-y-9-3-3+4 \\
	& \left(y^2-4 y+4\right)-3\left(x^2+2 x+1\right)-3\left(z^2-2 z+1\right)=-24 \\
	& \frac{x^2+2 x+1}{8}+\frac{z^2-2 z+1}{8}-\frac{y^2-4 y+4}{24}=1 \\
	& \frac{(x+1)^2}{8}+\frac{(z-1)^2}{8}-\frac{(y-2)^2}{24}=1
\end{aligned}
$$
) Esto es un hiperboloide de re reolución de una
: hoja con centro en $(-1,2,1)$ y con semiejes
$$
a=8 b b=24 \text { y } c=8
$$
1.4.i3iHallar el cértice del paraboloide eliptico
$$
\begin{aligned}
	& 4 x^2+3 y^2-12 z-8 x+5 y-5=0 \\
	& 4\left(x^2-2 x+1\right)+3\left(y^2+2 y+1\right)=15+4+3+12 z \\
	& \left.4(x-1)^2+3(y+1)^2=12 y+1\right)
\end{aligned}
$$

El uértice es $(1,-1,-1)$
1.4. 14 Hallar la ecuación de un parabol dide dévertice el punto $0,0,9)$ y que pasapor los puntos $(2,-3,2)$ y $(-6,-6,4)$ y curo
eje de sime tria es el ele y: colucionnsustitunchdo las coor de nadas
solucions
Solución sustituy nido las coor de nadas
$$
A x^2+C z^2=B Y\binom{4 A+4 C=-3 B}{36 A+16 C=-36 B}
$$
Escaneado con CamScanner


%%%%%%%%%%%%%%%%%%%%%%%%%%%%%%%%%%%%%%%%%


Despejando $A$ y $C$ en Función de B
$$
\begin{aligned}
	36 A+36 C & =-27 B \\
	-(36 A+16 C & =-36 B) \\
	\hline 0 \quad 20 C & =9 B
\end{aligned}
$$
$$
C=\frac{9}{20} B
$$
sustituy endo ${ }^{20} e n(1)$
Sustituyendo eri
$$
\begin{aligned}
	& 4 A+4\left(\frac{9}{20} B\right)=-3 B \\
	& 4 A+\frac{36}{20} B=-3 B \\
	& 4 A \neq-3 B-\frac{36}{20} B \\
	& 4 A=-\frac{56}{20} B
\end{aligned}
$$
$$
\begin{aligned}
	& A x^2+C z^2=B Y \\
	& -\frac{7}{10} B x^2+\frac{9}{20} B z^2=B Y
\end{aligned}
$$
multiplicando por $\frac{20}{B}$
$$
\begin{aligned}
	& -14 x^2+9 z^2=20 y \\
	& 9 z^2-14 x^2=20 y
\end{aligned}
$$ que seria la ervación hso
$A=-\frac{14}{20} B \quad A=-\frac{7}{10} B$ del paraboloide eliptico.
Escaneado con CamScanner


%%%%%%%%%%%%%%%%%%%%%%%%%%%%%%%%%%%%%%%%%


$$
\text { PROBIEMAS PROPUESTOS } 1.4
$$
1.4.1 2dentifiqueloss lugares aeométricos alagaue de ine 1 as figurentes pecueriones ytiazarla gráfica correspondiemte a cada función.
a) $8 x^2-2 y^2+50 z^2=0$
b) $4 y^2-x^2+25 z^2=100$

Solila superticie er un fone
ellptico uue tienecfi,
yome tie de sime. yepico que de simofica.
soli la superciciesun simetrla es el eje'X
sol: a superficle es elipice Hiperbotolde eliptice de 2 miantac cuyo ele de y
d) $5 y^2+15 z^2=20 x$ cuyo eje de simetria es el $x$
e) $6 y^2-18 z^2=22 x$
Soli la superficie es un paraturide Hiper 60 diro las seccinges tuans eerralesentos plane
$z=k$,
on
1,4.2. Ide intifique la spperficie que tiene laeciacin dada:
a) $9 x^2-5 y^2+45 z^2=45$
solstiper bolorde diphro decominga.
b) $6 x^2-3 z^2+2 y=0$
Sol: Patábolorde lliparéctico
c) $2 y^2-20 x^2=40$
solicilindro Hiperbotico
14.3. En los siguentes merisoritace la qraficadela
a) $5 x^2+9 y^2+z^2=36$
sel: Elypsoide
b) $5 x^2+20 y^2-z^2=100$
1Atpercoloude Eliptio
c) $x^2=y^2-z^2$
di Cono Eliptivo
d) $\frac{x^2}{36}+\frac{z^2}{25}=41$
salpar
Escaneado con CarnScanner


%%%%%%%%%%%%%%%%%%%%%%%%%%%%%%%%%%%%%%%%%


14.4 Encada uno de; as siquientes incisos, discuiale ycopstruyase el hiperboloide cuya ecuacion se da.
a) $\frac{x^2}{1}-\frac{y^2}{4}-\frac{z^2}{9}=1$ Dedos hojas
b) $x^2+y^2-2 z^2+16=0$ De una solahoja
c) $2 x^2-y^2+8 z^2+18=0$ de una solahoja
1.4.5: Hallase renentificar la ecuación del lugar geome trico de on punto que se muece de tal maneraque la sima de 10 suadrados de sus a 4 tancias a los ejes $x$ y y es slempleigual
$$
\text { Sol: } x^2+y^2+2 z^2=4
$$
1.46: En cál culo difè rencral se de maestra que el ciolumen limitgdo por un elisolde esigeqli 4 is(apc), siendp at b y c.los semifitinisueleolumen limita do porel e varorce :
$2 x^2+3 y^2+4 z^2-8 x+12 y-4=0$
Soli $96 \pi U^3$
1.4.7: Habler la ecuacjón del Heperboloide de tecrolo- cion de una sola zojade finidas pof la rotación de la tecta $y=4, z=x$ en fomno al eje z, constport la superticu:
$$
\text { Sol: } x^2+y^2-z^2=16
$$
64.8 Hallar el lugar geométrico de los Boptor equidistan del peano xy y del pusto $-1,2,-5$ ) Sol. $x^2+y^2+2 x-4 y+5 z+14=0$ Soli $3 x^2+4 y^2+4 z^2-32 x+16 y-8 z+84=0$
Escaneado con CamScanner


%%%%%%%%%%%%%%%%%%%%%%%%%%%%%%%%%%%%%%%%%


Unidad
II otras sistemas
de $\qquad$
$\omega$ yg eometra, a de mas de usar reqursas semo en pras tacions yotackon ordinadieatas funda se usa con die cuencra el cantego dé coorainad rectangulates a polates, filindrecas yesferices Y sobie to doentre estas mismasjec decir polaies a cilindricas, cilindricas a espemicas, etcetera.
2.t Coordenadas polquas. un punto $P$ delespaciorden figura 2.1.1) respectuamente soin ( $p, \alpha, \beta, \gamma$ ) siendo $B$ la aistantia aldrigen, odel sperma de courderigéas op y $\alpha, \beta, y$ y los and 105 de dirección de ap, las cela. - ciones ild das entre las coordernadis polsares y rectangulates esjá latumas por luj vestor pos que vien torels ascleamos a tener:
Praconuestre desectanigulares
$$
\begin{aligned}
	& \text { apolates: } \\
	& \operatorname{cosen}=\alpha=\frac{x}{x^2+y^2+z^2}=|\hat{\rho}| \\
	& \operatorname{cosen} \alpha=\frac{x}{\sqrt{x^2+y^2+z^2}} \\
	& \cos =\frac{y}{\sqrt{x^2+y^2+z^2}}
\end{aligned}
$$
$$
\operatorname{cosen} 0 \gamma^{\prime}=\frac{z}{\sqrt{x^2+y^2+z^2}}
$$

Existe una rdentidadigue se cumple en todas las andata polares:
$$
\left[\operatorname{cosen}^2 \alpha+\operatorname{cosen}^2 \beta+\operatorname{cosen} 0^2 \gamma^1=1\right.
$$
para conceestir de polases a coordenadas
$[x=\rho \operatorname{cosemo\alpha }] \mid y$ - $\operatorname{\rho roseno} \beta \mid$ $\square$
Escaneado con CamScanner


%%%%%%%%%%%%%%%%%%%%%%%%%%%%%%%%%%%%%%%%%


Elempla Resuettos:
2.1.1 Hatharara coordenadas polarpos de los puntos sigulentesal $0,3,4)$ (b) $(1,2,-2) c j(6,3,2)$
$$
\begin{aligned}
	& \text { Solccón; } y=3 \text { e: } z=4 \\
	& x=0, \sqrt{0+3^2+4^2}=\sqrt{0+9+16}=\sqrt{25}=5 \\
	& \text { coseno } \alpha=\frac{0}{5}=0 \quad \alpha=\operatorname{arco} \text { coseno } 0=90^{\circ} \\
	& \text { coseno } B=\frac{3}{5}=0.6 \quad \beta=\operatorname{arccoseno} 0.6=53.13^{\circ} \\
	& \text { coseno } \gamma=\frac{4}{5}=0.8 \lambda=\operatorname{arcocsen} 0.8=36.86^{\circ} \\
	& \text { sol: }\left(5,90^{\circ}, 53.13^{\circ}, 36.86^{\circ}\right)
\end{aligned}
$$
b)
$$
\begin{aligned}
	& x=1 \quad y=2 ; z=-2 \\
	& p=\sqrt{1^2+2^2+(-2)^2}=\sqrt{1+4+4}=\sqrt{9}=.3
\end{aligned}
$$
coseno $\alpha=\frac{1}{3} \quad \alpha=\operatorname{arcocoseno} \frac{1}{3}=70.53^{\circ}$
$$
\begin{gathered}
	\operatorname{cosen} 0=\frac{2}{3} \quad \beta=\operatorname{arcocosen} \\
	\operatorname{cosen} \theta=\frac{-2}{3} \quad 4=\operatorname{arcocos} \frac{-2}{3}=131.81 \\
	\text { sol: }(3,70.53,48,18, \sqrt{3} 1.81)^3
\end{gathered}
$$
c)
$$
\begin{aligned}
	& x=6 / y=3, z=2 \\
	& P=\sqrt{6^2+3^2+2^2}=\sqrt{36+9+4}=149
\end{aligned}
$$
$\operatorname{cosen} 0 \alpha=\frac{6}{7} \quad \alpha=\operatorname{arcocosen} \frac{6}{7}=316$
$$
\begin{aligned}
	& \operatorname{coseno} \beta=\frac{3}{7} \quad \alpha=\arccos \frac{3}{7}=670 \\
	& \operatorname{cosono} \phi=\frac{2}{7} \quad \beta=\operatorname{arcmon}-\frac{7}{\gamma}=73.39 \\
	& \operatorname{sol}\left(7,31.00,69.62^{\circ}\right.
\end{aligned}
$$
Escaneado con CamScanner


%%%%%%%%%%%%%%%%%%%%%%%%%%%%%%%%%%%%%%%%%


2,2
Coofderiadas cilindricas
Este sistemar es una extensión para el espacio
tridimensional delar coordenadar polares piraelp tridimensionad delar coordenador polares pira el plane.
Enutsistema de coordenalas cindricajun \$1: - (r, e) es uno representación polar de la priyección 2, - i es es la dis planoxy
Paraconverfir coordenaghes, rectangulases en cam ercacigneíen las caordenadia pi

Para conceechr de cilindicas a kectangulares
$$
x=r \operatorname{coseh} 0 \theta, y=r \operatorname{sen} \theta \theta, z=z
$$
para, conceertir de rectaryubes
a cilindricas:
$r^2=x^2+y^2$, Tangente $\theta=\frac{y}{x}, z=z$
elpunto $(0,0,0)$ es el polo,
comola repesentacion' como la repfesen aciom.
deunpuntoenel sistema
de codrdenados polares no es unca, se sigue tue que fampocopa que que famin de un punto
represisistema de coordenatas
es unica.
Figura 2.2 .1 esunica.

Eje mplo 2.2 .1 (oncererta et punto $(r, \theta, z)=\left(5, \frac{2 \pi}{3}, 12\right)$
a coordenadas tectangulases: a coordenadas rectangulares:
Solycióniusando las eruaciones de cqnuersión cilindricas a rectangulases, se abtiene:
$$
\begin{aligned}
	& x=5 \cos \frac{2 \pi}{3}=5\left(-\frac{1}{2}\right)=-2 \cdot 5=-\frac{5}{2} \\
	& y=5 \operatorname{sen} 0 \frac{2 \pi}{3}=5 \frac{\sqrt{3}}{2}=5(0,8660)=4,33 \\
	& z=12
\end{aligned}
$$
Escaneado con CamScanner


%%%%%%%%%%%%%%%%%%%%%%%%%%%%%%%%%%%%%%%%%


Por 10 tonto en coordepadas bectangulaver elpunto es $(x, y, z)=\left(-\frac{5}{2}, 4,333,12\right)$
Ejemplo 2.22 conuerpra pl punto $(x, y, z)=$
$(4,3,1)$ a coordenadas cilindricas..
solucigínia
usaingoblas ecvaciones para la concequsión
recangulaves a cilindricas, se obbiene
$$
\begin{aligned}
	& r=\sqrt{4^2+3^2}=\sqrt{16+9}=\sqrt{25}=5 \\
	& \operatorname{Tan} \theta=\frac{3}{4} \Rightarrow \theta=\operatorname{arcotan} \frac{3}{4}+n \pi=36.86+n \pi \\
	& z=2
\end{aligned}
$$
setieve dos opciones pasal y una cantidad infinita de opciones para t, dos representacioves conceenientes para este punto son.
$(5,36.86,1)$ ) $>0$ y $\theta$ en elcuadrante $I$ $(-5,216.86,1)$ + $j 0$ ye enel cuadrante II
las coondeñadas cilindricas sonecispecialmente a ge cyadas para la repres endarcion alu ryerticies cindyluas 1 de supfrficies dimeleplucionentas que el eye z se a ef be de simetra come se muestraen la figura $2,1,3$ a y $2,1,3 b$
$$
\begin{array}{ll}
	x^2+y^2 z & x^2+y^2=z^2 \\
	\text { parabooide } & \text { cono } \\
	\text { Figurazaza } &
\end{array}
$$
$$
\begin{aligned}
	& x^2+y^2-z^2=1 \\
	& r^2=z^2+1
\end{aligned}
$$

Hyper Golonde
Escaneado con CarnScanner
![](https://cdn.mathpix.com/cropped/2024_09_01_8d590a146b1d6bb9fa99g.jpg?height=181&width=458&top_left_y=492&top_left_x=41)


%%%%%%%%%%%%%%%%%%%%%%%%%%%%%%%%%%%%%%%%%



losplanos uerticale
contienen ale contrenen al ela ze y os planol horigontales heven coordenalas in, ndricas como semvestaerta figura $2.2,3$ a y 1 a flgura 2.2 .3 b.
$$
x^2+y^2=1
$$

Figura 2.2 .2 b)
Gamplo 2.2.3 Gonceersión Le coordengdas rectanqu. - lares a ceonde eada cimbiar. coordenzads cil indricas para.
cadqunadel us superficies
dadas en coordenas rectan-
5 Figura $223 a$
go) $\left.x^2+y^2=9 z^2 b\right) y^2=x$. Solución: Di chas superficies se muestiancomo epude

Fi4urci 2.2.zh)
Escaneado con CamScanner


%%%%%%%%%%%%%%%%%%%%%%%%%%%%%%%%%%%%%%%%%


figura 2. $[$ J/4 b) Figura 2,2,2 b)
a.) Solución: la gráfica de $x^2+y^2=9 z^2$ es un cono de se aprecia en la fiqura 22,4, a) alsust tu $x^2 f y^2=r$ es esta ecuación se conureste en coordetisulas cilindicas
$$
\begin{aligned}
	x^2+y^2 & =9 z^2 \leftarrow \text { Evación rectanqular } \\
	y^2 & =9 z^2 \leftarrow \text { Ecuacion cilindrica }
\end{aligned}
$$
b). La qráfira de la superficre $x^2=$ y es un cilindro plas bollco curas sertas generaticei ronga sushtweind. ${ }^2$ porkicosin y y por rseno ase cilindricas $x^2=$ y eciación retaryolar enadas
$r^2 \cos ^2 \theta=r \operatorname{sen} \theta \theta$ sustifyendo $x$ gor rcoseno $t$ y
por rseno por rseno a $r \operatorname{coseno}-r \operatorname{sen} \theta)=0$
$r \cos ^2 \theta-\operatorname{seno} \theta=0$. $r \cos ^2 \theta-\operatorname{seno\theta }=0$ Divicsito ca da lada de $r=\frac{\operatorname{sen} \theta \theta}{\cos ^2 \theta}$ despejando $r$
Escaneado con CamScanner
![](https://cdn.mathpix.com/cropped/2024_09_01_88bbbaf5407af42e26ffg.jpg?height=327&width=594&top_left_y=78&top_left_x=0)


%%%%%%%%%%%%%%%%%%%%%%%%%%%%%%%%%%%%%%%%%


$r=\frac{\operatorname{sen} \theta \theta}{\operatorname{cosen} \theta} \cdot \frac{1}{\operatorname{cosen} \theta}=\operatorname{secante} \theta$ tangente $\theta$
$r=\operatorname{secante} \theta$ tanyente $\theta$ Ecuación cilindrica
obsercee que existe un punto en el cual $t=0, a s i$ que al dividir entie r ambos miembras le tae evadion, no se ha perdidona da.
la forapersión de coordenadas rectangulaves a coordenoda clp coordenada muctio mas sencilla que la fonmerjion - i vlardes comos selierdras a cos a cooplepedarectancjemplo 2.2.4i conuesion cléloordenales ciling: \& coordeniadas lectangulates. Encveinte una ecvacian en coofdenadas bectangulares pafalosuperfice lepresenjada por la ecuacion
Solveión:
$y^2 \operatorname{cosen} 0^2 2 \theta+z^2+9=0$ écuación alindricia
$\operatorname{como} \operatorname{cosen} \theta 2 \theta=\operatorname{cosen} \theta^2 \theta-\operatorname{sen}^2 \theta$
$r^2\left(\operatorname{cosen}^2 \theta-\operatorname{sen}^2 \theta\right)+z^2+q=0$ Appls cando ledentidad
$\gamma^2 \operatorname{cosen}^2 \theta-\gamma^2 \operatorname{sen}^2 \theta+z^2+9=0$ trigonometrica
$x^2-y^2+z^2+9=0 \quad$ sustituyendo
$x^2-y^2+z^2=-9 \quad$ multiplicondo par $(-1)$
$y^2-x^2-z^2=9$
sustituyendo
multpliccondo por (-1)
iz figura 2.2 .5
$-y$


%%%%%%%%%%%%%%%%%%%%%%%%%%%%%%%%%%%%%%%%%


2,3 Coordenadas esféricas a is tahiial a seguhda yll la tercera soorde nada son ani los lós en las coordénadas que feu th
2 podemos neer en las coordenadas que or intizan
para la havegacion en los a viones por ejemplo: $\left(7500,-60^{\circ}, 45^{\circ}\right.$ ) és to es un radio de 7500 Km ,
elque diga $-60^{\circ} \mathrm{mm}$ plica $60^{\circ}$ en sentido opuest elque diga $-60^{\circ} \mathrm{mp}$ plica $60^{\circ}$ en sentido opuesto elque diga $-60^{\circ} \mathrm{mplica} 60$ en 45 implica $45^{\circ}$
a las manecillas del teloj el polo niqude, esto
ha cia a bajo. a pirtongitud, latitud yaltud
es son medidas de long es son medidaide longitud, latitud yaloitud, por ejemplo: (Altitud, longitud y latitud) 2.3.1 sistèmas de cooruenadas esfé́icas punfo penelespacio, se representa me didate Una triada ordenada ( $p, \theta, \phi$ )
1.- $\rho$ (iho) es la distancia de Pal origen, $P \geq 0$
2. - Ges el mismo ásigulo que se emplea en tas
-
3.- $\phi$ es el anqulo entree el ete z positico z+ y obserue que la primira y la tercera coordenadas f y $\phi$, son no vegatioas, pes la letra griega minus culartho y os la letra griega minuscula phi Acontinuación presentamos una delación entre las coordenadas rectangulases y coordenadar espénas parapasar de. uno a ofrosutema se usan las syumentes

Esféricas a rectangulares
$$
x=\rho \operatorname{sen} o \phi \cos \theta, y=\beta \operatorname{sen} \phi \operatorname{sen} \theta \theta, z=\operatorname{cosen} \phi \phi
$$

Rectangulares a esférica's
$$
p=x^2+y^2+z^2 \text {, tangente } \theta=\frac{y}{x}, \phi=\arccos \left(\frac{z}{\sqrt{x^2+y^2+z^2}}\right)
$$
Escaneado con CarnScanner


%%%%%%%%%%%%%%%%%%%%%%%%%%%%%%%%%%%%%%%%%


Pasainaliaar un cambio entre coordenadas
cas dricas coordenadas espóscas, se uan (v) Pas ecuaciones siguientesi

Esféricas a cillindricas ( $r \geq 0$ ) $r^2=\rho^2 \operatorname{sen}^2 \phi,: \theta=\theta, z=p \operatorname{cosen} \theta \phi$
cllindricas a esféricas (r $\geqslant 0$ ):
$$
\begin{aligned}
	& P=\sqrt{r^2+z^2,} \theta=\theta, \phi=\operatorname{arccosen} \theta\left(\frac{z}{\sqrt{r^2+z^2}}\right) \\
	& \text { Higura } 2.2 .6 \quad r=\rho \operatorname{sen} \theta \phi=\sqrt{x^2+y^2}
\end{aligned}
$$
is te sis tema de roor tienadas és úthl parasu-
Escaneado con CamScanner


%%%%%%%%%%%%%%%%%%%%%%%%%%%%%%%%%%%%%%%%%


Ejemploiconuersión de coordenadas rer tangu-
-lares a coorde nadas éféricas
Encuentre una ecuación en coorderiadas eveéricas parae la superficie repres en tida por ca da un a de las ecuaciones.rectangulare's siguientes:
a) Cono: $x^2+y^2=z^2$ b) Esfera $x^2+y^2+z^2-4 z=0$
solución sustifuyendo $x, y, y z$ tenemos lo sigurente:
$$
x^2+y^2=z^2
$$
$\rho^2 \operatorname{sen} \sigma^2 \phi \operatorname{cosen}^2 \theta+\rho^2 \operatorname{sen}^2 \phi \operatorname{sen} 0^2 \theta=\rho_{\operatorname{cosen}^2}^2 \phi$.
$\rho^2 \operatorname{sen}^2 \phi(\underbrace{\operatorname{cosen}^2 \theta+\operatorname{sen}^2 \theta}_{\rho^2 t})=\rho^2 \cos ^2 n 0^2 \phi$
$$
\rho^2 \operatorname{sen}^2 \phi=\rho^2 \cos ^2 \phi \Rightarrow \frac{\operatorname{sen}^2 \phi}{\operatorname{cosen}^2 \phi}=1 \quad \rho \geq 0
$$
$\tan ^2 \phi=1 \quad \phi=\frac{\pi}{4} \quad 0^{\prime} \phi=\frac{3 \pi}{4}$
b) como $p^2=x^2+y^2+z^2$ y $z=\rho \cos \phi$, la ecuación tiene la siguiente forma esférica:
$$
\rho^2-4 \rho \operatorname{cosen} 0 \phi=0 \Rightarrow \rho(\rho-4 \cos \phi)=0
$$

Descastando la posibiliday te que $\rho=0$, se tiewe la ecuación esférica:
$$
p-\operatorname{cosenc} \phi=0 \quad 0^{\prime} p=4 \operatorname{cosen} .
$$
porlotanto el con unto so ción ua
- aser la esfera quese puede wer en la figura 2.2 .6
Escaneado con CarmScanner


%%%%%%%%%%%%%%%%%%%%%%%%%%%%%%%%%%%%%%%%%




Ejercicios propuisitos unidad II
2.1: Hallar las cepordenadas polaresi cilndricashy sotier $(1,2,2$. de punto, cuyas coordenadas ectangulates
Respueta;
Cbordenadas polares $P=\sqrt{1^2+2^2+2^2}=\sqrt{9}=3$
$$
\begin{gathered}
	\alpha=\operatorname{arcocoseno} \frac{1}{3}=70^{\circ} 32^{\prime} \quad \beta=\operatorname{arcocoseno} \frac{2}{3}=48.11^{\circ} \\
	\gamma=\operatorname{arco} \operatorname{coseno} \frac{2}{3}=48.11^{\circ}
\end{gathered}
$$
sol: $\left(3,70^{\circ}, 32^{\prime}, 48^{\circ}, 11^b, 48^{\circ}, 11^{\circ}\right)$
coordenadas alíndricas $\rho=\sqrt{\sqrt{2}^2+2^2}=\sqrt{5}$
$\theta=\operatorname{arcotan} \frac{y}{x}=\operatorname{arcotan} \frac{2}{1}=63.345^{\circ} ; z=2$ sol $(\sqrt{5}, 6.3935,2)$
Coordenadas esféricas ; $\rho=\sqrt{1^2+2^2+2^2}=\sqrt{9}=\sqrt{3}$
$\theta=\operatorname{arcotang} \frac{y}{x}=\operatorname{arcotang} \frac{2}{1}=63.435^{\circ}$,
こ $\phi=\operatorname{arcocosen} \frac{\frac{z}{p}}{\rho}=\operatorname{arcocosen} \frac{\pi}{3}=43^{\circ} \%^{\circ}$
Escaneado con CamScanner


%%%%%%%%%%%%%%%%%%%%%%%%%%%%%%%%%%%%%%%%%


2.2 Hallar las coordenadas rectangulares del punto cuyas coordenadas wlindricas som $(6,120,-2):$
$$
\text { Soli }(-3,3 \sqrt{3},-2)
$$
2,3 Hallar las coordenadas sectangulares del punto cuyas coordenadas esféricas son $\left(6,45^{\circ}, 30^{\circ}\right)$
Sol: $x=6$ seno $30^{\circ} \cos 45^{\circ}=6\left(\frac{1}{2}\right)\left(\frac{\sqrt{3}}{2}\right)=\frac{6 \sqrt{2}}{4}=\frac{3 \sqrt{2}}{2}=\frac{3}{2}=2.1213$
$$
\begin{aligned}
	& y=6 \operatorname{sen} 30^{\circ} \operatorname{sen} 45^{\circ}=6\left(\frac{1}{2}\right)\left(\frac{\sqrt{2}}{2}\right)=\frac{6 \sqrt{2}}{4}=\frac{3}{2} \sqrt{2}=\frac{3}{\sqrt{2}}=2 / 263 \\
	& z=6 \operatorname{cosen} 030^{\circ}=\frac{6 \sqrt{3}}{2}=3 \sqrt{3}
\end{aligned}
$$

Juego la soluciónes: $\left(\frac{2}{\sqrt{2}}, \frac{3}{\sqrt{3}}, 3 \sqrt{3}\right)$
2.4: Haliar las coordenadas rettangulares, polares yespéricas del punto cuyas coordenadas citindricas son ( $3,120,2$ )
Solución: par ias coordenadas rectangulares:
$$
501:\left(-\frac{3}{2}, \frac{3 \sqrt{3}}{2}, 2\right)
$$
$$
\begin{aligned}
	& x=3 \operatorname{coseno.12}=3(1 / 2)=-\frac{3}{2} \quad z=2 \\
	& y=3 \text { seno } 120^{\circ}=3\left(\frac{\sqrt{3}}{2}\right)=\frac{3 \sqrt{3}}{2} \\
	& \left(\frac{3 \sqrt{3}}{2}, 2\right)
\end{aligned}
$$

Para las coordeñedospolares:
$$
\begin{aligned}
	& \rho=\sqrt{\left(\frac{-3}{2}\right)^2+\left(\frac{3 \sqrt{3}}{2}\right),^2 2^2}=\sqrt{\frac{9}{4}+\frac{27}{4}+4}=\sqrt{\frac{9+27+18}{4}} \\
	& \rho=\sqrt{\frac{52}{4}}=\sqrt{13} \\
	& \alpha=\operatorname{arcocoseno}\left(\frac{3}{2 \sqrt{13}}\right)=114^{\circ} 5^{\prime} \quad \beta=\operatorname{arccosen}\left(\frac{3 \sqrt{3}}{2 \sqrt{13}}\right)=46^{\circ} 7^{\prime} \\
	& \gamma=\operatorname{arcocoseno} \frac{2}{\sqrt{13}}=56 \% 9^{\prime}
\end{aligned}
$$

Espéricas $\rho-\sqrt{13} \quad \theta=\operatorname{arcotarg}$ ente $\frac{3 \sqrt{3} / 2}{-3 / 2}=$ archangente $\frac{3 \sqrt{3}}{3}$
$\phi=\operatorname{arcocoseno} \frac{2}{\sqrt{13}}=56^{\circ} 9^{\prime} \quad \theta$ oarrotangente $-\sqrt{5}=120^{\circ}$
$$
\text { soli }\left(\sqrt{13}, 120^{\circ}, 56^{\circ} 91\right)
$$
Escaneado con CamScanner


%%%%%%%%%%%%%%%%%%%%%%%%%%%%%%%%%%%%%%%%%


2. 5 Expresar la ecuación $x^2+y^2+2 z^2-2 x-3 y-z+2=0$ en
coordenadas clindricas!

Sol: $\rho^2-\rho(2 \operatorname{cosen} 0 \theta+3 \operatorname{sen} 0 \theta)+2 z^2-z+2=0$
2.6.- Expresar la ecuación $2 x^2+3 y^2-6 z=0$ en coorde nadas esfericas:
Sol: $2 \rho \operatorname{sen} o^2 \phi \operatorname{cosen}^2 \theta+3 \rho \operatorname{sen} o^2 \phi \operatorname{sen} o^2 \theta-6 \operatorname{cosen} \theta \phi=0$
2. 7 Expresar la e cuacion $p+6 \operatorname{sen} \phi \operatorname{cosen} \theta \theta+4 \operatorname{seno} \phi \operatorname{sen} n \theta$ encoordenadas rectanguilares:
sol: $x^2+y^2+z^2+6 x+4 y-8 z=0$
2.8.- Ex presar la ecuación $z=\rho^2$ cosenoze en coordenadas rettangulases:
Sol: $z=x^2+y^2$
2.9.- Expresar la ecuación $x^2+y^2-z^2=36$ en coorde nadas polares!
soli $\rho^2\left(1-2 \operatorname{cosen}^2 \gamma^{\prime}\right)=36$
2.10 Expresar la ecuación escrita en coor denadaspolares, $y=\rho$ coseno $\alpha$ coseno $\beta$ en coordenadar yectangulares.
Sol: $z=x y$
2.11: Hallar las coordenadas polares de los pun tos sigurente. a) $(0,11) ; b)(0,-2,2) ; c)(1,-2,2) ; d)(6,3,2)$ e) $(8,-4,1$,

Sol: a) $\left(\sqrt{2}, 90^{\circ}, 45^{\circ}, 45^{\circ}\right)$
b) $\left(2 \sqrt{2}, 90^{\circ}, 135^{\circ}, 135^{\circ}\right)$
c) $\left(3, \operatorname{arcocosen} \frac{1}{3}, \operatorname{arcocosen}(-2 / 3)\right.$, arcocoseno $\left.2 / 3\right)$
d) $(7, \operatorname{arcocosen} 06 / 7, \operatorname{arcocosen}(3 / 4)$ arcacosen $02 / 7)$
e) $(9, \operatorname{arcocosen} 08 / 9, \operatorname{arcocosen}$ of $4 / 5), \operatorname{arccosen} 01 / 9)$
2.12: Hallar las coordenadas cilindricas de los pintor del

Sol: A $\left(1,90^{\circ}, 1\right)^{2,1}\left(b^{\prime}\right)\left(2,270^{\circ},-2\right)(c)(\sqrt{5}, 2 \pi$-arcoting $1 / 2,2)$
d) $(3 \sqrt{5}$, arcoking $1 / 2,2)$ e) $(4 \sqrt{5}, 2 \pi-3 n c t r y, 2)$
2.3 Hallar las coordenadas espersicas de los puntos
soli a) $\left(12,90^{\circ}, 45^{\circ}\right)$; b) $\left(2 \sqrt{2}, 270^{\circ} ; 135^{\circ}\right)$ ic) $(3,2 \pi$-arcotany $(2)$, arocoeno 2 d) $\left(7, \operatorname{arcotany}\left(\frac{1}{2}\right), \operatorname{arcocosen} 0 \frac{3}{7}\right)$ e) $\left.\left(\frac{1}{2}, 2 \pi-\operatorname{arctang}\left(\frac{1}{2}\right), \operatorname{arcocosen} \frac{1}{9}\right)^3\right)$
Escaneado con CarmScanner


%%%%%%%%%%%%%%%%%%%%%%%%%%%%%%%%%%%%%%%%%


2.14 Hallar las coorde nadas rectanqu lates de los puntos
e) $\left(2,45^{\circ}, 120^{\circ},-60^{\circ}\right)$
sa) $\left.(0, \sqrt{3}, 1) ; b\left(\frac{3}{2}, \frac{3 \sqrt{2}}{2},-3 / 2\right) ; c\right)(-2,-2,-2 \sqrt{2})$
d) $(-3 \sqrt{3} / 2,3 / 2,0) ; 0)(\sqrt{2},-1,1 ;)$
2.15. Hallar las coordenadas vectangolases de los puntas
$\begin{aligned} & \text { b) }(1,330,-2)(4,45,2)(4)(8,120,3)(e)(6,30,-3) \\ &\text { Solución:a) }(-3,3 \sqrt{3},-2) ; b)(r 3 / 2,-1 / 2,-2) ; c)(2 \sqrt{2}, 2 \sqrt{2}, 2) \\ &d)(-4,4 \sqrt{3}, 3) \text { e }(3 \sqrt{3}, 3,-3) \\ &\end{aligned}$
2.16 Hallar las coordenadas eectangulases de los puntes cuyas coordenadas escerras son: a) $\left(4,210^{\circ}, 30^{\circ}\right)$
b) $\left(3,120^{\circ}, 240^{\circ}\right)$; c) $\left(6,330^{\circ}, 60^{\circ}\right)$ d) $\left(5,150^{\circ}, 210^{\circ}\right)$

Solucion: a) $(-\sqrt{3}, 1,2 \sqrt{3})$ b) $\left(\frac{3 \sqrt{3}}{4},-9 / 4,-3 / 2\right)$
)
c) $(9 / 2,-3 \sqrt{3} / 2)$
3.) d) $\left(\frac{5 \sqrt{3}}{4},-5 / 41\right.$
$-\frac{5 \sqrt{3}}{2}$
2.17. Hallar las coordenadlas esféricas de los puntes a) $\left.\left.(8,120 ; 6) ; b)\left(4,30^{\circ},-3\right) ; c\right)\left(6,135^{\circ}, 2\right), d\right)\left(3,150^{\circ}, 4\right)$ Solución: a) $\left(10,120^{\circ}, \arccos \frac{3}{5}\right) ;$ b) $\left(5,30 ; \operatorname{arcocoseno}\left(\frac{-3}{5}\right)\right)$
c) $\left.\left(2 \sqrt{10}, 135, \frac{\sqrt{10}}{10}\right) ; d\right)\left(5,150, \operatorname{arcocoseno} \frac{4}{5}\right)$
2.18.Ex presas ér coordenadas es éricasilas e cuaciones c) $\$ x+5 y-8 z=12$

Solvelón! a) $\theta=$ arcotangente $(-5 / 4)$ b) $5 p \operatorname{cosen} 0^2 \theta-$
b) $5 \rho \operatorname{cosen}^2 \theta-4 \cdot \rho \operatorname{sen} \theta^2 \theta+2 \operatorname{cosen} \theta \theta+3 \operatorname{sen} \theta \theta=0$
c) $p-8 \operatorname{cosen} 0 \theta=0$
2.19. PG das las ecuaciones siquientes, en coordenadar cilingricas naliar sunaturadeza expresarla en c) $\left.\rho^2+z=25 ; \mathcal{2}=2045 ; e\right) \rho^2-z^2=1$
Escaneado con CarmScanner


%%%%%%%%%%%%%%%%%%%%%%%%%%%%%%%%%%%%%%%%%


Solveión 2.19
Y a) $x^2+y^2+3 z^2=49$ Elipsoide derevolución
b) $x^2+y^2=b y \cdot$ ilindro circular recto
c) $x^2+y^2+z^2=16$ esfera
d) $y=x$ Plan 0
e) $x^2+y^2-z^2=1$; Hiperbolorde de una hoja.
2.20.- Expresar en coordenadas polaigs, las equacio-
- nespsigulenfe. $a x^2+y^2+y^2-a^2$
c) $2 x^2 73 y^2+2 z^2-6 x+2 y=0$ d) $z=2 x y$
chent
solveión 2.20:
a) $p\left(1-\operatorname{cosen} 0^2 x\right)+6 \cos \phi=0$
b) $p^2\left(1-2 \operatorname{cosen}^2 \theta^1\right)=a^2$
c) $\rho\left(2+\operatorname{cosen} 0^2 \beta\right)-6 \operatorname{cosen} 0 \theta+2 \operatorname{cosen} 0 \beta=0$
d) coseno $\mu=2 \rho$ coseno $\alpha \operatorname{cosen} 0 \beta$
Escaneado con CamScanner


%%%%%%%%%%%%%%%%%%%%%%%%%%%%%%%%%%%%%%%%%


3.1 Funciones de Wariakle nectorlal y ecuaciones parminetcas devnafunclón.

Un liector representa una magntuduce tiene direción sepuede denotnolor tiene un prinipio jun aut qú He ferencilenotar uaino una dislancia, a unipunto de recordemoisa dustancia siempie lea a serpositiea huncg caminamby al heves, simos smple prente quene anudia wvelta it toshanos caminanido haclael. Fripte es por ésto que un weftor, al que denoaneres H(t) jel cesil denotaremos como:
$H(t)=x(t), \dot{0}+y(z) j \in n$ elpho
$$
r(t)=x(t) i+y(t) J+z(t) k \varepsilon_n e l \text { espaclo }
$$
para estos dos uertars, wanos a considesian esa
"listariciala la quellamakimes moduke en Aunoiom de las cootderadas:
modiflece aluecton ban cicamplat de forman lnfintes mat de tal manura quese puedein adecuar a ef cadar cenomginos cons el mikimento do una pajicícía lo large de una cus lea, is cono xLt y ₹(4) $12 a$, an con respecio a ए?, \&.i.nbl= de calculo difelencial elntaval, is thedename
guei Eina funcion de la ferma $x(t)=t(t), Y(t)=y(t)$
$z(t)=h(t)$
Defienición de la funcióncectorial.
Una funcrón de la forma
$r(t)=f(t) i+g(t) J$ bana el plano.
$0 \mid(t)=f(t) i f g(t) i t(t) k$ issuat!
Escaneado con CamScanner
$$
\begin{aligned}
	& 1+(t) \mid=\sqrt{x(t-1]^2+[y(t)]^2} \text { Ens elplanc } \\
	& |f(t)|=\sqrt{[x(t)]^2+[x(t)]^2} \text { En el espercio }
\end{aligned}
$$


%%%%%%%%%%%%%%%%%%%%%%%%%%%%%%%%%%%%%%%%%


Es unatunción wectorial, dondo las funciones compowentes fig yh son funclove al partmetro ts el uvál conic ya se dyo anteriormente se aclppto para aplicm flerraciones como inteyración, dorivén yeualuacion dal limite. al iqual gue losceegloves las unciongs uertoriales se puedes de notarde la sigurente fomm.
$r(t)=\langle f(t), g(t)\rangle$ parer el plano,0blen
$$
r(t)=\langle f(t), g(t), h(t)\rangle \text { para el espacro }
$$

Vina uriup enel plano ó en el erpacio jertó tormada par, an numero incinito de puptas de hecho dos, curuas difieventes puedentener la misma graficialio podemo wer con las siguientes funciones:
siguientes funciones: $r(t)=\operatorname{senti}+$ cosenot $J$
las cuales podemos $\quad, \quad$ y
tabulad:
\begin{tabular}{|c|c|c|c|c|}
	\hline$t$ & $\operatorname{sen} 0$ & $\operatorname{cosen} t$ & $V_1(t) \| i_2(t)$ & 1 \\
	\hline 0 & 0 & 1 & 1 & 1 \\
	\hline$\pi / 3$ & $1 / 2$ & $\sqrt{3} / 2$ & 1 & 1 \\
	\hline$\pi / 4$ & $\frac{1}{\sqrt{2}}$ & $\frac{1}{\sqrt{2}}$ & 1 & 1 \\
	\hline$\frac{\pi}{2}$ & 1 & 0 & 1 & 1 \\
	\hline$\frac{3 \pi}{4}$ & $\frac{1}{\sqrt{2}}$ & $-\frac{1}{\sqrt{2}}$ & 1 & 1 \\
	\hline$\frac{\pi}{2 \pi}$ & 0 & -1 & 1 & 1 \\
	\hline$\frac{5 \pi}{4}$ & $-\frac{1}{\sqrt{2}}$ & $\frac{1}{\sqrt{2}}$ & 1 & 1 \\
	\hline$\frac{3 \pi}{2}$ & -1 & 0 & 1 & 1 \\
	\hline$\frac{7 \pi}{4}$ & $-1 / \sqrt{2}$ & $\frac{1}{\sqrt{2}}$ & 1 & 1 \\
	\hline$\frac{2 \pi}{}$ & 0 & 1 & 1 & 1 \\
	\hline
\end{tabular}
$y_2(t)=\operatorname{sen}^2 t i+\operatorname{cosen} 0^2 t j$ $r_1(t)=0(t) \quad r_i(t)=0^2\left(t 1^2\right)$ $r_1(t)=1 \frac{1}{2} i \pm \frac{1}{2} J \quad r_2(t)=\frac{3}{4} \frac{1}{4}+\frac{1}{4} J$
$$
r_1(t)=\frac{1}{\sqrt{2}} i+\frac{L}{2} j r_2(t)=\frac{1}{2} i+\frac{i}{2} j
$$
$>$
$$
\hat{y}
$$
bsto es la jrácica de. es tas de curadsise
tabla 3.1 uni hairio.
Escaneado con CamScanner


%%%%%%%%%%%%%%%%%%%%%%%%%%%%%%%%%%%%%%%%%


is claro entonces que cada kariable f(t), $g(t) y$ h(t) serian funciones de vna waslable parame - trica t, que representa un numero eal como sepudo ceer en la tabla 3.1 y como funciones podemos considerar que fiejen undominio y un contradominio, suje tandose a lo contemplado paste unal-uncon que séhabia wisto en cursos. anteriores. se vsa la letra t para de notar a la uoria ble independientejen frimer ug ar por que ya se, hapia nitudaconanderior tica dee plano yila ininea secta y porque incearia la plica alonés de Wa fisica ca les como desplasamiento lee la siclaidy a deleracion que son uunciones uectarialer debenson, parametrica ties básica. puesto aue nos permite dericear intearas y apiicas el concepto del Limite de una unción escalar a una coy a base son los wedtares, asimismo es limppitaite elconsiderar que para una funcion enien plano: $r(t)=-f(t) i+g(t) j \dot{1}$ el dominco de estas dos cunciones debeéstar definido paro quelo pueda estar zड́ ceqd $\gamma(t)$ y para ef espario $-f(t)=f(t) i+g(t) j+h(t) k$, deben estar definidas esas tres funciones para que r(t) 10 este.
\&jemplo 3.3 .1 , sea la funcioin $r(t)=\left\langle t^2, 1\right.$ in $\left.t, \frac{1}{t}\right\rangle$ El dominio que tendría estafunción seria
- $t>0$ ya que int ni 1 ejtas deliniclas para parat $=0$, qunque $t^2$ i la isté pant todos los cealoves reales, basta conque una de las
Escaneado con CamScanner


%%%%%%%%%%%%%%%%%%%%%%%%%%%%%%%%%%%%%%%%%


Unflonis, no estó deflinida pasang poder aplican el cancepto defunción en olvector.
$3,3 \div$
El limite le una Auncios̀ lectorial $+(t)$, se detine foinamdo los limite de sus Aonciowes

Silemṕre r cuandalos lemites de tas funcioves componentes, existan.
Enierminas generoles, se podfía hecer una también de la dexinicion e-s viamads también tagurosia del concepto de
Los uimites de las tunciones uectorioles obe decen a las misma
reglas que los limites de funciones con balores Hales. Eemplo 3.2. Deterume limr(t) dondea
$$
\gamma(t)=\left(e^{-2 t}\right) i \neq \frac{t^2}{\cos 2 t} j+\cos ^2 t K
$$

Solución, De acuerdo con la defrnición:
$\operatorname{Lim}_{t \rightarrow a} x(t)=L$ asi de esta forma tendeíamos que el limite de res el vector uvas compqneiztes son los limites de las cuncrones componerites der:
$$
\begin{aligned}
	\operatorname{Lim}_{t \rightarrow 0} r(t) & =\left[\lim _{t \rightarrow 0}\left(e^{-2 t}\right)\right] i+\left[\lim _{t \rightarrow 0} \frac{t^2}{\operatorname{los} 2 t}\right] j+\left[\lim _{t \rightarrow 0} \cos t\right] k \\
	& =[1] i+0 j+[1] k=i+k
\end{aligned}
$$
* también las identidade trigonométricas en elfoniepto de limite, sybuse habiant istudado $\lim _{t \rightarrow 0} \frac{\text { senet }}{t}=1$, se pieder apticar entasturitones aectoriales
Escaneado con CamScanner
$$
\begin{aligned}
	& 5 i \gamma(t)=\langle f(t), g(t), h(t)\rangle, 1 \text { legoi } \\
	& \lim _{t \rightarrow a} x(t)=\left\langle\lim _{t \rightarrow a} f(t), \lim _{t \rightarrow 2} g(t), \lim _{t \rightarrow a} h(t)\right\rangle
\end{aligned}
$$


%%%%%%%%%%%%%%%%%%%%%%%%%%%%%%%%%%%%%%%%%


Una función wectorial $t$ es continua en $a, s$ s
$$
\lim _{t \rightarrow a} r(t)=r(a)
$$

En wista de la de finicion, po demas wer que res continua en a si, y solosi, sus funtiones componeites t, g. yh son con znuas en qu us aseit contipuach un in tervalo $a$ Eiemplo 3.3" Halear el limite de la siguiente unción:
$$
\lim _{t \rightarrow \infty}\left\langle\frac{1-t^2}{1+t^2} \text {, arcocotangentet } t, \frac{1+t^2+2 t^3}{3^2+t^3}\right\rangle
$$

Solución $\lim _{t \rightarrow \infty}\left\langle\frac{1 t^2-t^2}{1 / t^2+t^2 t^2}\right.$, arcotangente1 $\left.\frac{1 / t^3+t^2 t^3+2 t^3 / t^3}{3 t / t^3+t^3 / t^3}\right\rangle$
$$
\left\langle-\frac{1}{1}, 0, \frac{0+0+2}{0+1}\right\rangle=\langle-1,0,2\rangle
$$
3,3.2 curuas en el espacio
Existe una reldaion eitrecho entre las funciones cectoriales continuas y las curleas en el espacio. vamos a considerar que las funciones fig y h son funciowes continuas cin walores reales eí u'n interualo Irentoncesel conjuito c de todos la definida por la curcea, vamos a tener:
$$
x=f(t), \quad y=g(t), \quad z=h(t)
$$
y siendo t una variable paramétrica a lo largo
del interealo I se liasha una curcea en ef espacio, las ecuaciones (1), se llamant ecuaciones paramétricas de C.
Ejemplo 3.2.1, Describa la curcea defitiola por la función ceectorial:
$$
r(t)=\langle 2+t, 3-5 t, 4+8 t\rangle
$$
solución las ecuaciones paramétricar corres pondrentes ban a ses las siguientes,
$$
x=2 t, t, y=3-5 t, z=4+8 t
$$
queserín las ecuaciowes paramétricas
y pasalela alal pasa por el punto $(2,3,4)$
Escaneado con CamScanner


%%%%%%%%%%%%%%%%%%%%%%%%%%%%%%%%%%%%%%%%%


las corceas en el plano, tarnbrén. pueden depiesen-tarse por so notación eectorial.as i po driamos definur, para la curca dada por las ecuaciones
$\nu$ Pandinetricas: $x=t^2+2 t+1$ y $y=t+1$, poderia
describirse mediante la notación bectorial
$$
r(t)=\left\langle t^2+2 t+1, t+1\right\rangle=\left(t^2+2 t+1\right) i+(t+1) J^{\prime}
$$
nótese que $x=y^2$ que en el plano de coondenadas repjesentaria la parábola $Y=\sqrt{X}$ que sepodría definir de la siguiente forma:
Figura 3.2.1 De esta forma podemor liearr,
a la grafica ue una funcion
cectorial a travez delarela-cion ente sus ceariables.

Ejemplo 3.2.3. Trace la coriea
tuya ecuaciom uectorial es tuya ecuaciom uectorial
$r(t)=\operatorname{cosen} o t i+\operatorname{sen} 0 t_j+t{ }_K$ Enes te caso las ecuaciowos paramé tricas de la circea cean ases:

Figura 3,3.?
$x=\operatorname{cosen} 2 t ; y=\operatorname{sen} 0 t ; z=t$


%%%%%%%%%%%%%%%%%%%%%%%%%%%%%%%%%%%%%%%%%


Eceimplo 3.3.3. Determine und ecuacióm bectorialt यa de to i pana ê ćaquénto de rectanglee uectoriats. ar, esta dada por la elva croin
$$
r(t)=(1-t) r_0+t r_1 0 \leq t \leq 1
$$
asi tomanclo $r_0=\langle 2,5,7\rangle$ y $\sqrt{1}, \in\langle 3,2,5\rangle$ Lye go para una equación uectorial del segmento
$$
\begin{aligned}
	& r(t)=(1-t)\langle 2,5,7\rangle+t\langle 3,2,5\rangle \quad 0 \leq t \leq 1 \\
	& r(t)=\langle 2+t, 5-3 t, 7-2 t\rangle \quad 0 \leq t \leq 1
\end{aligned}
$$

Sas equacionesparamétricas do esto logar geonitio
sepueden definircomo.
$$
x=2+t ; \quad y=5-3 t ; \quad z=7-2 t \quad 0 \leqslant t \leqslant 1
$$
3.4Dericeada de una función vectorlal

La deriseada de una función gectorial, se define a partie
do un imies síniar al de la le dericade de una funcon
de ealores reales de calores reales
-datinición. Si r(t) es unq funcjón zeectorlal la derisa - eectoriacon respecto a porse define comola funciona
$$
r^{\prime}(t)=\lim _{h \rightarrow 0} \frac{r(t+h)-r(t)}{h}(\text { (. })
$$

Eldominio de ri(t) consiste en todar la caslares det
eneldominio de $r(t)$ paralos que el limito existe. la deriuacta de r(t) pue de expres shor é tomo mp
Escaneado con CamScanner


%%%%%%%%%%%%%%%%%%%%%%%%%%%%%%%%%%%%%%%%%


Es importante senalar que r'ti) uss ungecfor, no un nemero es calar y por cons invente tiéve dirección fara cada walor de t, Eexcepto sir $(t)=0$, paraesle caso, aunque suculor es 0 , no tiene una direcición especifical
3.4.1: Definición de las derineadas defuncroverbec-toriales para el plano y el espacio
Teorema.3. I: Deriuación de Funcioves ceectoriales
$$
r^{\prime}(t)=f^{\prime}(t) ;+g^{\prime}(t) j \text { enel plano }
$$
$$
r^{\prime}(t)-t^{\prime}(t) i+g^{\prime}(t) j+h(t) k \text { enel espacio }
$$

Ejemplo 3.4.1. Dericación de Funcroves leectoriales Parala función lee torlal dada por $r(t)=(t+1) i+\left(t^2+2 t+1\right)$ ) encontrar $r^{\prime}(t)$, bosque ar adenás la curcea plana representada por $r(t)$ y las gráficas de $t(1)$ y + $^{\prime}(1):$
Solveión: Dorieanelo ca da una de las compowentes $r^{\prime}(1)=i+(2(1)+2) j \quad \gamma^{\prime}(1)=i^{\prime}+4 j^{\prime}$
Parai $+(i)=(1+1) i+\left(i^2+2(1)+1\right) j$
$$
=2 i+4 j
$$

La jrá́ica pluna de la iuslea La ceamos a definir a partir. de $x=t+1 \quad y=t^2+2 t+1$
bien sería la parnibola $y=x^2$ comose pue de wer en la (1.90. a \# 3.31
Escaneado con CamScanner


%%%%%%%%%%%%%%%%%%%%%%%%%%%%%%%%%%%%%%%%%


3.42 Reglas de Derileación trovema 3.3 Q 1
sean urciai dos funciowes ceertoriales doriuables, An un escalas y unafuncion concealores reates,
$\theta-\frac{d}{d t}[u(t)+v(t)]=u^{\prime}(t)+v(t)$
$21-\frac{d}{d t}[c u(t)]=c v^{\prime}(t)$
3: $-\frac{d}{d t}[f(t) u(t)]=f^{\prime}(t) u(t)+f(t) u^{\prime}(t)$
4. - $\frac{d}{d t}[u(t) \cdot v(t)]=u^{\prime}(t) \cdot v(t)+u(t) \cdot v^{\prime}(t)$
$5,-\frac{d}{d t}[u(t) \times v(t)]=u^{\prime}(t) \times v(t)+u^{\prime}(t) \times v^{\prime}(t)$
$6_1-\frac{d}{d t}\left[u(f(t)]=f^{\prime}(t) u^{\prime}(f(t))\right.$ Regla de la
$7,-S_1^i f(t) \cdot f(t)=c$, entonces $f^{\prime}(t)-f^{\prime}(t)=0$
Ejemplo 3.42. Aplicación de las propiedades Para las funciones ceectoriales de firidas por $r(t)=5 t i+4 j+\ln t K \quad y u(t)=2 t^2 i-5 j+8 k$
hallar a) $D_t\left[r(t) \cdot u(t)\right.$ b) $D_t\left[u(t) \times u^{\prime}(t)\right]$
a) Como $r^{\prime}(t)=5 i+0 j+\frac{1}{t} \quad$ y $u^{\prime}(t)=4 t i+0 j+0 k$

Utiliz ando la fórpula(4)
$$
\frac{d}{d t}[r(t) \cdot u(t)]=r(t) \cdot u(t)+r(t) \cdot u(t)=\left\langle 5 i+0 j+\frac{k}{t}\right\rangle\left\langle 2 t{ }^2 5 j+8 k\right\rangle+
$$
$$
10 t^2+\frac{8}{t}+20 t^2=30 t^2+\frac{8}{t}
$$
$<5+i+4 j+1 n t 5<\langle 4 t i+0 ;+01$
Escaneado con CamScanner


%%%%%%%%%%%%%%%%%%%%%%%%%%%%%%%%%%%%%%%%%


b)
$$
\begin{aligned}
	& u(t)=2 t^2 i-5 j+8 k \quad y^{\prime \prime}(t)=4 i \\
	& u^{\prime}(t)=4 t i+0 j+0 k
\end{aligned}
$$
63

Asi tendría mos,que aplican do la
Rormv a
3,5 Laintegral definida de una cunción veectonal contmo $Y(t)$ pae de dejinse gasilgeal que fas funcione con calojes veakes, saluo que la integral es on uector, pero entonces se pue de qapresar la integral de r on terminos de lap integgales de sus funciones componestes $f, g_n y h$ comosigue:
Escaneado con CamScanner
$$
\begin{aligned}
	& D_t\left[u(t) \times u^{\prime}(t)\right]=\left[u(t) \times u^{\prime \prime}(t)\right]+\left[u^{\prime}(t) \times u^{\prime}(t)\right] \\
	& =\left|\begin{array}{ccc}
		i & j & k \\
		2 t^2 & -5 & 8 \\
		4 & 0 & 0 \\
		i & j & k \\
		2 t^2 & -5 & 8
	\end{array}\right|+\left|\begin{array}{ccc}
		i & k & k \\
		4 t & 0 & 0 \\
		4 \neq 0 & 0 \\
		i & j & k \\
		4+0 & 0
	\end{array}\right| \\
	& 6+0+32 \dot{j}-20 k+\overrightarrow{0}+0+0 \\
	& 32 j-20 K
\end{aligned}
$$
$$
\begin{aligned}
	& \int_a^b r(t) d t=\lim _{n \rightarrow \infty} \sum_{i=1}^n r(t t) \Delta t \\
	& \text { yportanto } \\
	& \left.=\lim _{n \rightarrow \infty}\left[\sum_{i=1}^n n(t) \Delta t\right) i+\left(\sum_{i=1}^n g\left(t_i^*\right) \Delta t\right) j+\left(\sum_{i=1}^n h\left(t_i^*\right) \Delta t\right) k\right] \\
	& \left.\int_a^b c t\right) d t=\left(\int_a^k(t) d t\right) i+\left(\int_a^b(t) d t\right) j+\left(\int_a^h(t) d t\right) k
\end{aligned}
$$


%%%%%%%%%%%%%%%%%%%%%%%%%%%%%%%%%%%%%%%%%


Es to quies de cir que podemar elea/vinela integral de, una funcion coectorial integrando cada función componente.
Asi ceamos a tever 10 siguiente:
Definición de la integral de ona función leectural 1. $-\sin (t)=f(t) i+g(t) f$ donde $-y g$ son continuas en $[a, b]$, en tonces lainte- grap indefinidalóantiderreada)de
$\int \delta(t) d t=\left[\int f(t) d t\right] i+\left[\int g(t) d t\right] j$ En elplano y su integral definida en el mercealo $a \leq t \leq b, e s:$
$\int_a^b r(t) d t=\left[\int_i^b f(t) d t\right]_i+\left[\int_j \dot{g}(t)\right.$. En el plano $2,-s i \gamma(t)=f(t) i+g(t) j+h(t) K$ donde $f i g$ y $h$ son continuas en $[a, b]$ ] entonces la initegral in defini da Co antide ricada) de estí.
$$
\begin{aligned}
	& \text { Co antide siceada } \\
	& \int r(t) d t=\left[\int f(t) d t\right] i+\left[\int g(t) d t\right]_{-}+\left[\int h(t) d t\right] K
\end{aligned}
$$
para el espacio
y suintegral definidacie el intercalo $a \leq t \leq b$ es
$$
\int_a^b r(t) d t=\left[\int_a^b(t) d t\right] i+\left[\int_d^b g(t) d t\right] K t\left[\int_a^b h(t) d t\right] k
$$
Escaneado con CamScanner


%%%%%%%%%%%%%%%%%%%%%%%%%%%%%%%%%%%%%%%%%


la antidericada de una función rectorial es una familia de funciones ceectoriles)
te: $\square$
auya diterencia entresi es un ceector constante C, de tal manera que si $r(t)$ esuna función uectorial tricimensional $\int r(t) d t$, se obtjenentres constantes diferenter deintegracion,
$\int f(t) d t=F(t)+C_1 \cdot \int g(t) d t=G(t)+C_2$
$$
\int h(t) d t=H(t)+C_3
$$
donde $F^{\prime}(t)=f(t), G^{\prime}(t)=g(t)$ y $\left.H^{\prime}(t)=h(t) \rightarrow\right)$ Estas tres constantes es calares forman Un ceector como constante de integración
$$
\begin{aligned}
	\int r(t) d t & =\left[F(t)+C_7\right] i+\left[G(t)+C_2\right] J+\left[H(t)+C_3\right] \\
	& =[F(t) i+G(t) \dot{\dot{j}}+H(t) k]+\left[C_1 I+C_2 J+C_3 k\right] \\
	& =R(t)+e
\end{aligned}
$$
donde $R^{\prime}(t)=r(t)$
Eje mplo 3.5.1 integración de una función
Hallar la integral in definida
$$
\int\left(t^2 i+2 t j+6 k\right) d t .
$$
Escaneado con CamScanner


%%%%%%%%%%%%%%%%%%%%%%%%%%%%%%%%%%%%%%%%%


Solugióni integrando componento por componente: $\int\left(t^2 i+2 t j+6 k\right) d t=\frac{t^3}{3} i+2 \frac{t^2}{2} j+6 t k+' C$
Proce deriamos de la mismamanera para una
incion en plano, enel siguiente siqmplo je
muestar como, muesta como, ellalvar ta, integral definida de una función lee ctorlal:
Elemplo 3.5.2 ziteqral definida unafunción Elealvar la integral
$$
\begin{aligned}
	\int_0^2\left(e^{2 t i} ; e^{-t}+t k\right) d t & =\frac{e^{2 t}}{2} i-e^{-\frac{t}{j}}+\left.\frac{t^2}{2} k\right|_0 ^2 \\
	& =\frac{e^4}{2} i-e^{-2}+\frac{2^2}{2} k \\
	& =(27,3 i-0,1353 j+2 k)-(0,5 i-j+0 k) \\
	& =26,8 i+0,864 j+2 k
\end{aligned}
$$

Gomplo 3.5.3: Eualuar la integral:
$$
\int_{\frac{\pi}{6}}\left[(\sec t \tan t) i+\left(\sec ^2 t\right)^{\prime}+(2 \operatorname{sen} t \cdot \cos t) k\right] d t:
$$

Secante $\frac{i}{3} i+$ tangente $t+\frac{\operatorname{tecseno} 2}{2} \mathrm{~K} \frac{\frac{\pi}{4}}{2}$
(Secante $\left.\frac{\pi}{4} i+\operatorname{tangente} \frac{\pi}{4} J+\frac{2}{2} \pi / 2 K\right) \Rightarrow$
secante $\left.\frac{\pi}{6} i+\tan g e n t e \frac{\pi}{6} j+\frac{\operatorname{coseno} \pi / 3}{2} K\right)$
Escaneado con CamScanner
$$
\begin{aligned}
	& (\sqrt{2} i+j+O K)-\left(\frac{2}{\sqrt{3}} i+\frac{1}{\sqrt{3}} i+0.25 k\right) \\
	& \left(\sqrt{2}-\frac{2}{\sqrt{3}}\right) i+\left(1-\frac{1}{\sqrt{3}}\right) j \neq 0.25 k \\
	& \left(\frac{\sqrt{6}-2}{\sqrt{3}}\right) i+\left(\frac{\sqrt{3}-1}{\sqrt{3}}\right)^i-0.25 k=0.26 i+0.4226 J-0.25 k
\end{aligned}
$$


%%%%%%%%%%%%%%%%%%%%%%%%%%%%%%%%%%%%%%%%%


Ejemplo, 3.5.4 La primifiela de una, unción cectoral Hallar la primitiea de la función.
$$
r(t)=t \operatorname{sen} t i+t \cos t \dot{j}+\frac{1}{1+t^2} k
$$
)
Que satisface la condl ción:t(0)=3i-2i+K
Solución: $r(t)=\int_0 r^{\prime}(t) d t$
asi tendriamosque
$$
=\left(\int_{0 \leq} t \operatorname{sen} t d t\right) i+\left(\int t \cos t d t\right) \dot{\pi}+\left(\int \frac{d t}{1+t^2}\right) K
$$
$\left(\operatorname{sen} t-t \cos t+C_1\right) i+\left(\cos t+t \operatorname{sen} t+C_2\right) J^{\prime}+\left(\operatorname{arcotan} t+C_3\right)$ hacrendo $t=0$, usando el hecho de que;
$r(0)=3 i-2 j+K$, se tiene:
$$
\begin{aligned}
	r(0) & =\left(\operatorname{sen} 00-0(1)+C_1\right) i+\left(\cos 0+0(0)+c_2\right) j+\left(0+c_3\right) K \\
	& =\left(0+0+C_1\right) i+\left(170+(2) j+\left(0+c_3\right) k=3 i-2 i+k\right.
\end{aligned}
$$
asi uamos a tener $c_1 i+\left(1+C_2\right) j+C_3 K=3 i-2 j+K$ $C_1=3 \quad 1+C_2=-2 \quad C_3=1 \Rightarrow c_1=3 \quad C_2=-3 \quad C_3=1$
asitendriamo que la funcion original seric $r t)=(\operatorname{sen} t-t \cos t+3) i+(\cos t+t \operatorname{sen} t-3) j+(\operatorname{arcotant+1}) R$ Problemas propuestas 3,5
1.- Bosquerela yráfica de /a curiea representado por la funcia
$$
r(t)=2 \cos t i+2 \operatorname{sen} t \dot{j}+t K
$$
2.- Bosquejt; la curea lepresentada porla función
$$
r(t)=t i+t^2 j+\frac{3 t}{2} k
$$
3.- tualve el limitei $\lim _{t \rightarrow 0}\left(t^2 i+3 t_j+\frac{-\cos t}{t} k\right)$ Sol: $0 i+O I+0 K$
Escaneado con CamScanner


%%%%%%%%%%%%%%%%%%%%%%%%%%%%%%%%%%%%%%%%%


(
4i-\&ricuentre $r(i)$ para $r(t)=\langle t$ sent, $t \cos t, t\rangle$ solpcioin $=\langle t \cos t+\operatorname{sen} t,-t \operatorname{sen} t+\cos t, 1\rangle$ S.- Encuentre el caector de posición rsi
$$
\left.r^{\prime}(t)=4 e^{2 t}\right)+3 e^t r^{\prime} y(0)=2
$$
solvolón: $r(t)=\left(2 e^{2 t}\right) 1+\left(3 e^t-3\right) j=\left(2 e^{2 t}\right) i^{\circ}+3\left(e^t-1\right) 5^{\prime}$ 6i- Ewálv'e la integral definida: $\int_0^1(8 t$ it $t J-k) d t$ Solución: $4 i+\frac{1}{2} j-k$
3.6 Iongitud de Areo y Cureatura
teosema, Longitud de arco de una curceas en el espacio. la longitud de arco de cenesteintercealo es
0
$$
s=\int_c^b \sqrt{\left[x^{\prime}(t)\right]^2+\left[y^{\prime}(t)\right]^2+\left[z^{\prime}(t)\right]^2} d t
$$

Ecemplo: Encuentre la longitud de la hélice circliar: $r(t)=3 \cos t \dot{i}+3 \operatorname{sen} t \dot{+}+4 t k$ Enel in tenealo $[0,2 \pi]$ : es decir $0 \leqslant t \leqslant 2 \pi$ Solución: la grá fica es una hélice circular de radio 2\% lee ase la figura:
O.
\begin{tabular}{|c|c|c|c|}
	\hline$t$ & $x(t)=3 \cos t$ & $y(t)=3 \operatorname{sen} t$ & $z(t)=4 t$ \\
	\hline 0 & 3 & 0 & 0 \\
	\hline$\pi / 4$ & $3 / \sqrt{2}$ & $3 / \sqrt{2}$ & $\pi$ \\
	\hline$\pi / 2$ & 0 & 3 & $2 \pi$ \\
	\hline$\pi / 4$ & $-3 \sqrt{2}$ & $3 \sqrt{2}$ & $3 \pi$ \\
	\hline$\pi$ & -3 & 0 & $4 \pi$ \\
	\hline$\pi / 4$ & $-3 / \sqrt{2}$ & $-3 \sqrt{2}$ & $5 \pi$ \\
	\hline $3 / 2$ & 0 & -3 & $6 \pi$ \\
	\hline$\pi / 4$ & $3 / \sqrt{2}$ & $-3 \sqrt{2}$ & $7 \pi$ \\
	\hline $2 \pi$ & 3 & 0 & $8 \pi$ \\
	\hline
\end{tabular}
Escaneado con CamScanner


%%%%%%%%%%%%%%%%%%%%%%%%%%%%%%%%%%%%%%%%%


Escaneado con CamScanner


%%%%%%%%%%%%%%%%%%%%%%%%%%%%%%%%%%%%%%%%%



 
